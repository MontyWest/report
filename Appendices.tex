%___________________________________________
%*************************************************************
% Appendices
%___________________________________________
%^^^^^^^^^^^^^^^^^^^^^^^^^^^^^^^^^^^^^^^^^^^^^^^^^^^


\begin{appendices}

\setcounter{table}{0}
\renewcommand\thetable{\thesection\arabic{table}}
\setcounter{figure}{0}
\renewcommand\thefigure{\thesection\arabic{figure}}
\setcounter{lstlisting}{0}
\renewcommand\thelstlisting{\thesection\arabic{lstlisting}}

\captionsetup{font={footnotesize,it}}

\addtocontents{toc}{\protect\setcounter{tocdepth}{2}}
\makeatletter
\addtocontents{toc}{%
  \begingroup
  \let\protect\l@section\protect\l@subsection
  \let\protect\l@subsection\protect\l@subsubsection
}
\makeatother

\section{Perceptions}
\label{app:percond}

\vspace{-1\baselineskip}
\begin{table}[h]
  \begin{center} \footnotesize
    \begin{tabular}{ l c c c}
    & \textbf{0} & \textbf{1} & \textbf{2} \\
    \textbf{MarioMode} (MM) & small & big & fire \\
    \textbf{JumpAvailable} (JA) & false & true & -  \\
    \textbf{OnGround} (OG) & false & true & -  \\
    \textbf{EnemyLeft} (EL) & false & true & -  \\
    \textbf{EnemyUpperRight} (EUR) & false & true & -  \\
    \textbf{EnemyLowerRight} (ELR) & false & true & -  \\
    \textbf{ObstacleAhead} (OA) & false & true & -  \\
    \textbf{PitAhead} (PA) & none & close & far \\
    \textbf{PitBelow} (PB) & false & true & -  \\
    \textbf{MovingX} (MX) & none & left & right \\
    \textbf{MovingY} (MY) & none & down & up \\
    \end{tabular}
    \end{center}
  \caption{Displays the different perceptions and what their byte value represents.}
  \label{tab:AgKey}
\end{table}

\begin{minipage}{0.9\linewidth}
\begin{lstlisting}[language=scala, basicstyle=\scriptsize\ttfamily, caption={Pit detection in the perceptions classes.}, label={lst:prcptpit}]
// Returns NONE if there are no pits ahead, FAR and CLOSE if there is one in a certain number of columns relative to mario.
case object PitAhead extends BytePerception(7, 2) {
    val NONE: Byte = 0; val CLOSE: Byte = 1; val FAR: Byte = 2;
    val COL_CLOSE_L = 1; val COL_CLOSE_R = 1; val COL_FAR_L = 2; val COL_FAR_R = 2
    def apply(environment: Environment): Byte = {
        val pitOp: Option[Int] = 
          Perception.getOpens(environment, COL_CLOSE_L, COL_FAR_R).headOption
        pitOp match {
            case Some(x) if (x >= COL_CLOSE_L && x <= COL_CLOSE_R) => CLOSE
            case Some(x) if (x >= COL_FAR_L && x <= COL_FAR_R) => FAR
            case _ => NONE
       }
    }
}

def getOpens(environment: Environment, a: Int, b: Int): List[Int] = {
    val level = environment.getLevelSceneObservationZ(2);
    val test = (x: Byte) => x == 0 || x == GeneralizerLevelScene.COIN_ANIM;
    val bottomRow = level.length - 1
    val mario = getMarioPos(environment)
    val left = max(0, min(a, b) + mario._2)
    val right = min(level(0).length, max(a, b) + mario._2)   
    var opens: List[Int] = left to right toList
    
    for {
        i <- mario._1 + 1 to bottomRow
        if (!opens.isEmpty)
    }{
        opens = opens.filter { j => level(i)(j) == 0 || level(i)(j) == GeneralizerLevelScene.COIN_ANIM }
    }
    opens.map { x => x - mario._2 }
}
\end{lstlisting}
\end{minipage}

\section{Handcrafted Agent Rulesets}
\label{app:har}
\setcounter{table}{0}

Included below are the full rulesets for the project's handcrafted agents. Blank entries denote a {\footnotesize DONT\_CARE} condition or a {\footnotesize FALSE} action.
\vspace{\baselineskip}

\begin{table}[h]
  \begin{adjustwidth}{-2cm}{-2cm}
  \begin{center} \scriptsize
    \begin{tabular}{ | c | c | c | c | c | c | c | c | c | c | c | c || c | c | c | c |}
    \hline
    \multirow{2}{*}{\textbf{\#}} & \multicolumn{11}{c ||}{\textbf{Conditions}} & \multicolumn{4}{c |}{\textbf{Actions}} \Tstrut \\ \cline{2-16}
	& \tiny MM & \tiny JA & \tiny OG & \tiny EL & \tiny EUR & \tiny ELR & \tiny OA & \tiny PA & \tiny PB & \tiny MX & \tiny MY & \tiny Left~ & \tiny Right & \tiny Jump~ & \tiny Speed \TBstrut \\ \thickhline
	1 & & 0 & 1 & & & & & & & & 	& & T & & \\ \hline
	- & & & & & & & & & & & 		& & T & T & T \\ \hline
    \end{tabular}
  \end{center}
  \end{adjustwidth}
  \caption{Ruleset for handcrafted Forward Jumping Agent.}
  \label{tab:FJA}
\end{table}

\begin{table}[h]
  \begin{adjustwidth}{-2cm}{-2cm}
  \begin{center} \scriptsize
    \begin{tabular}{ | c | c | c | c | c | c | c | c | c | c | c | c || c | c | c | c |}
    \hline
    \multirow{2}{*}{\textbf{\#}} & \multicolumn{11}{c ||}{\textbf{Conditions}} & \multicolumn{4}{c |}{\textbf{Actions}} \Tstrut \\ \cline{2-16}
	& \tiny MM & \tiny JA & \tiny OG & \tiny EL & \tiny EUR & \tiny ELR & \tiny OA & \tiny PA & \tiny PB & \tiny MX & \tiny MY & \tiny Left~ & \tiny Right & \tiny Jump~ & \tiny Speed \TBstrut \\ \thickhline
	1 & & 0 & 1 & & & & & & & & 	& & T & & \\ \hline
	2 & & & & & & 1 & & & & & 	& & T & T & \\ \hline
	3 & & & & & & & 1 & & & & 	& & T & T & \\ \hline
	4 & & & & & & & & 1 & & & 	& & T & T & T \\ \hline
	5 & & & & & & & & & 1 & & 	& & T & T & T \\ \hline
	- & & & & & & & & & & & 		& & T & & \\ \hline
    \end{tabular}
  \end{center}
  \end{adjustwidth}
  \caption{Ruleset for handcrafted Simple Reactive Agent.}
  \label{tab:SRA}
\end{table}

\begin{table}[!h]
  \begin{adjustwidth}{-2cm}{-2cm}
  \begin{center} \scriptsize
    \begin{tabular}{ | c | c | c | c | c | c | c | c | c | c | c | c || c | c | c | c |}
    \hline
    \multirow{2}{*}{\textbf{\#}} & \multicolumn{11}{c ||}{\textbf{Conditions}} & \multicolumn{4}{c |}{\textbf{Actions}} \Tstrut \\ \cline{2-16}
    
	& \tiny MM & \tiny JA & \tiny OG & \tiny EL & \tiny EUR & \tiny ELR & \tiny OA & \tiny PA & \tiny PB & \tiny MX & \tiny MY & \tiny Left~ & \tiny Right & \tiny Jump~ & \tiny Speed \TBstrut \\ \thickhline
	1 & & & & & & & & & 1 & & 		& & T & T & T \\ \hline
	2 & & 1 & 1 & & & & 1 & 2 & 0 & & 		& & T & T & \\ \hline
	3 & & 0 & 0 & & & & 1 & 2 & 0 & & 		& & T & & \\ \hline
	4 & & 1 & 1 & & & & 0 & 2 & 0 & & 		& & T & & T \\ \hline
	5 & & 1 & 1 & & & & 0 & 1 & 0 & 2 & 		& & T & T & T \\ \hline
	6 & & 0 & 1 & & & & & 1 & 0 & 2 & 		& & T & & T \\ \hline
	7 & & 0 & 0 & & & & & 2 & 0 & 2 & 		& T & & & \\ \hline
	8 & & 0 & 0 & & & & & 1 & 0 & 2 & 1 		& T & & & T \\ \hline
	9 & & 0 & 0 & & & & & 1 & 0 & & 		& & T & T & T \\ \hline
	10 & & & 0 & & & 1 & & 0 & 0 & 2 & 1 		& T & & & T \\ \hline
	11 & & 0 & 1 & & & 1 & & 0 & 0 & & 		& & T & & \\ \hline
	12 & & & & & & 1 & & 0 & 0 & & 		& & T & T & \\ \hline
	13 & & & 0 & & & 1 & & 0 & 0 & & 		& & T & T & T \\ \hline
	14 & & & & & 1 & & 1 & 0 & 0 & & 		& T & & T & \\ \hline
	15 & & 0 & 1 & & & & 1 & 0 & 0 & & 		& & T & & \\ \hline
	16 & & & & & & & 1 & 0 & 0 & & 		& & T & T & T \\ \hline
	17 & & & & & & & 1 & 0 & 0 & 2 & 		& & T & T & \\ \hline
	18 & & & 1 & & & & & & & & 		& & T & & T \\ \hline
	
	-  & & & & & & & & & & & 		& & T & & \\ \hline
    \end{tabular}
  \end{center}
  \end{adjustwidth}
  \caption{Ruleset for handcrafted Complex Agent.}
  \label{tab:CA}
\end{table}

\section{Full Level and Evaluation Options}
\label{app:loem}
\setcounter{lstlisting}{0}

\begin{minipage}{0.9\linewidth}
\begin{lstlisting}[language=scala, basicstyle=\scriptsize\ttfamily, caption={Full field definitions for MW\-Level\-Options and MW\-Evaluation\-Multipliers described in Section \ref{subsec:paramclasses}}]
/***
 * Options that control level generation
 */
class MWLevelOptions(
    val blocks: Boolean,   // Blocks appear
    val cannons: Boolean,  // Cannons appear
    val coins: Boolean,    // Coins appear
    val deadEnds: Boolean, // Dead ends appear in terrain forcing Mario to turn back
    val enemies: Boolean,  // Enemies/Creatures appear
    val flatLevel: Boolean, // Level is flat, no change in elevation
    val frozenCreatures: Boolean, // All creatures don't move
    val pits: Boolean,     // Pits appear
    val hiddenBlocks: Boolean, // Hidden blocks appear
    val tubes: Boolean,    // Tubes/Pipes appear
    val ladders: Boolean,  // Ladders appear
    val levelDifficulty: Int, // Difficulty of level, effective range 0-25, 0 easiest
    val levelLength: Int,  // Length of level in blocks
    val levelType: Int,    // Type of level, 0-Outside 1-Cave, 2-Castle
    val startingMarioMode: Int, // Mode Mario starts as 0-small, 1-big, 2-fire
    val timeLimit: Int     // Number of Mario seconds allowed to complete level
)

/***
 * Multipliers for several level playing statistics.
 * Comments describe the statistic. For example, if Mario
 * completes the level It will be win * 1, otherwise it will win * 0
 */
class MWEvaluationMultipliers(
    val distance: Int, // Distance travelled by Mario in pixels (16 pixels to a block)
    val win: Int,      // 1 for level complete, 0 otherwise
    val mode: Int,     // Mario's final mode on completion or death, 2-fire, 1-big, 0-small
    val coins: Int,    // Number of coins collected
    val flowerFire: Int, // Number fire flowers collected
    val kills: Int,    // Number of enemy kills
    val killedByFire: Int, // Number of kills by fireball
    val killedByShell: Int, // Number of kills by shell
    val killedByStomp: Int, // Number of kills by stomp
    val mushroom: Int, // Number of mushrooms collected
    val timeLeft: Int, // Mario seconds left on completion, 0 if level not completed
    val hiddenBlock: Int, // Number of hidden blocks hit
    val greenMushroom: Int, // Number of green mushrooms collected
    val stomp: Int   // Unused
)
\end{lstlisting}
\end{minipage}

\section{Comparator Task Options}
\label{app:comptask}
\setcounter{lstlisting}{0}


\begin{minipage}{0.9\linewidth}
\begin{lstlisting}[language=scala, basicstyle=\scriptsize\ttfamily, caption=Parameter classes for the comparator task described in Section \ref{subsec:comptask}]
val defaultEvaluationMultipliers = new MWEvaluationMultipliers(
        	1,    //Distance
        	2048, //Win
		16,   //Mode
		16,   //Coins
		64,   //FlowerFire
		58,   //Mushroom
		42,   //Kills
		4,    //KilledByFire
		17,   //KilledByShell
		12,   //KilledByStomp
		8,    //TimeLeft
		24,   //HiddenBlock
		58,   //GreenMushroom
		10)   //Stomp
		
val compBaseOptions: MWLevelOptions = new MWLevelOptions(
		true,  //blocks
		true,  //cannons
		true,  //coins
		false, //deadEnds
		true,  //enemies
		false, //flatLevel
		false, //frozenCreatures
		true,  //gaps
		false, //hiddenBlocks
		false, //tubes
		false, //ladders
		0,     //levelDifficulty
		256,   //levelLength
		0,     //levelType
		2,     //startingMarioMode
		200)   //timeLimit

def compUpdate(levelSeed: Int): (Int, MWLevelOptions) => MWLevelOptions = (i: Int, options: MWLevelOptions) => {
    options.withLevelLength(
             ((((i+levelSeed) * 431) % (501+levelSeed) ) % 462) + 50)
           .withTimeLimit((options.levelLength * 0.7).toInt)
           .withLevelType(i % 3)
           .withLevelDifficulty((compNumberOfLevels - i)/32)
           .withPits(i % 4 != 2)
           .withCannons(i % 6 == 2)
           .withTubes(i % 5 == 1)
           .withCoins(i % 5 != 0)
           .withBlocks(i % 6 != 2)
           .withLadders(i % 10 == 2)
           .withFrozenCreatures(i % 3 == 1)
           .withEnemies(!(i % 4 == 1))
           .withStartingMarioMode(
               if (i % 7 == 5 || i % 7 == 1) {
                 if (i % 2 == 0) 0 else 1
               } else 2)
}
\end{lstlisting}
\end{minipage}


\section{Final Learning Parameter File}
\label{app:paramfile}

\begin{minipage}{0.9\linewidth}
\begin{lstlisting}
parent.0 = @ec.es.ESDefaults es.params

#General
breedthreads 	= 4
evalthreads 	= 1
seed.0 = 1
seed.1 = 909
seed.2 = 499311
seed.3 = 90032

# +++ ES +++
breed = ec.es.MuPlusLambdaBreeder
es.mu.0 		= 5
es.lambda.0 	= 45
generations 	= 1000

# +++ POP +++
pop.subpops 						= 1
pop.subpop.0.size 					= 50
pop.subpop.0.species 				= com.montywest.marioai.learning.ec.vector.RulesetSpecies
pop.subpop.0.species.fitness		= ec.simple.SimpleFitness

#	 Rulelength 15, So 20 rules gives 300
pop.subpop.0.species.genome-size 	= 300
pop.subpop.0.species.ind 			= ec.vector.ByteVectorIndividual

#	These will be ignored, but warnings otherwise 
pop.subpop.0.species.min-gene		= -1
pop.subpop.0.species.max-gene		= 2
pop.subpop.0.species.mutation-type	= reset
pop.subpop.0.species.mutation-prob  = 0.0
pop.subpop.0.species.crossover-type = one

#	Handles min and max
pop.subpop.0.species.dynamic-param-class = com.montywest.marioai.learning.ec.params.RulesetParams

#	Condition params
pop.subpop.0.species.condition							= true
pop.subpop.0.species.condition.mutation-prob 			= 0.05
pop.subpop.0.species.condition.favour_byte 				= -1
pop.subpop.0.species.condition.favour_probability 		= 0.5

# MarioMode Condition
pop.subpop.0.species.condition.0.favour_probability 	= 0.95

# EnemyLeft Condition
pop.subpop.0.species.condition.3.favour_probability 	= 0.9
\end{lstlisting}
\end{minipage}

\begin{minipage}{0.9\linewidth}
\begin{lstlisting}
#	Action params
pop.subpop.0.species.action								= true
pop.subpop.0.species.action.mutation-prob 				= 0.09


# +++ STATS +++
stat.num-children 						= 1
stat.child.0 							= com.montywest.marioai.learning.ec.stats.RulesetEvolveStatistics
stat.child.0.gen-file					= ../lmm-lel-gen.stat
stat.child.0.final-file 				= ../lmm-lel-final.stat
stat.child.0.final-agent-file			= lmm-lel-final
stat.child.0.best-agent-file			= lmm-lel-best
stat.child.0.diff-agent-file			= lmm-lel-diff
stat.child.0.best-agent-limit			= 800
stat.child.0.diff-agent-limit			= 800


# +++ MUTATION +++
pop.subpop.0.species.pipe.source.0 		= ec.es.ESSelection
pop.subpop.0.species.pipe 				= com.montywest.marioai.learning.ec.vector.breed.RulesetMutationPipeline


# +++ EVAL +++
eval.problem							= com.montywest.marioai.learning.ec.eval.EvolvedAgentRulesetEvaluator

#	Seeds for generating levels
#	each generation g, seed used is:
#		prev_seed + add + g*mult
#   where prev_seed is seed_start on g  = 0
eval.problem.seed						= true
eval.problem.seed.start					= 3348
eval.problem.seed.add					= 284839
eval.problem.seed.mult					= 2568849

#	Evaluation Multiplier
eval.problem.mults						= true
eval.problem.mults.distance				= 1
eval.problem.mults.win					= 3200
eval.problem.mults.mode					= 200
eval.problem.mults.kills				= 100
eval.problem.mults.time-left			= 2

#	Fallback Action
eval.problem.fallback-action			= true
eval.problem.fallback-action.right		= true
eval.problem.fallback-action.speed		= true
\end{lstlisting}
\end{minipage}

\begin{minipage}{0.9\linewidth}
\begin{lstlisting}
#	Levels
eval.problem.level						= true
eval.problem.level.num-levels			= 10

eval.problem.level.base.dead-ends 		= false
eval.problem.level.base.enemies			= false
eval.problem.level.base.cannons			= false
eval.problem.level.base.pipes			= false
eval.problem.level.base.start-mode-num  = 2
eval.problem.level.base.length-num		= 200
eval.problem.level.base.time-limit		= 100
eval.problem.level.0.difficulty-num		= 2
eval.problem.level.0.type-num			= 1
eval.problem.level.1.difficulty-num		= 3
eval.problem.level.1.type-num			= 0
eval.problem.level.2.difficulty-num		= 5
eval.problem.level.2.type-num			= 2
eval.problem.level.3.difficulty-num		= 10
eval.problem.level.3.type-num			= 0
eval.problem.level.4.difficulty-num		= 2
eval.problem.level.4.flat				= true
eval.problem.level.5.difficulty-num		= 7
eval.problem.level.6.difficulty-num		= 2
eval.problem.level.6.flat				= false
eval.problem.level.6.enemies			= true
eval.problem.level.6.frozen-enemies		= true
eval.problem.level.7.difficulty-num		= 2
eval.problem.level.7.frozen-enemies		= false
eval.problem.level.8.difficulty-num		= 3
eval.problem.level.8.type-num			= 1
eval.problem.level.8.tubes				= true
eval.problem.level.9.difficulty-num		= 5
eval.problem.level.9.type-num			= 0
eval.problem.level.9.cannons			= true
eval.problem.level.9.blocks				= false
eval.problem.level.9.tubes				= false
\end{lstlisting}
\end{minipage}

\addtocontents{toc}{\endgroup}
\end{appendices}