\documentclass[a4paper, 11pt]{article}

%\usepackage[parfill]{parskip}
\usepackage{titlesec}
\newcommand{\sectionbreak}{\clearpage}
\usepackage{hyperref}
\usepackage{graphicx}
\usepackage{listings}
\usepackage[nounderscore]{syntax}
\usepackage{multirow}
\usepackage{chngpage}
\usepackage{array}
\usepackage{pbox}
\usepackage{amsmath}
\usepackage{color}


\definecolor{mygreen}{rgb}{0,0.6,0}
\definecolor{myred}{rgb}{0.6,0,0.298}
\definecolor{mymauve}{rgb}{0.58,0,0.82}
\definecolor{myorange}{rgb}{1,0.52,0}

\lstset{
	basicstyle=\footnotesize\ttfamily, 
	columns=fullflexible, 
	breaklines=true,
	captionpos=b,                    % sets the caption-position to bottom
  	commentstyle=\color{mygreen},    % comment style
  	escapeinside={\%*}{*)},          % if you want to add LaTeX within your code
  	keywordstyle=\color{myred},       % keyword style
  	stringstyle=\color{myorange},     % string literal style
	emphstyle=[1]{\color{blue}},
	showstringspaces=false
}

% "define" Scala
\lstdefinelanguage{scala}{
  emph=[1] {%
  	Int, Boolean, Vector, Option, Array, Byte, Some, None, Nil, Tuple2
  },
  morekeywords={abstract,case,catch,class,def,%
    do,else,extends,false,final,finally,%
    for,if,implicit,import,match,mixin,%
    new,null,object,override,package,%
    private,protected,requires,return,sealed,%
    super,this,throw,trait,true,try,%
    type,val,var,while,with,yield},
  otherkeywords={=>,<-,<\%,<:,>:,\#,@},
  sensitive=true,
  morecomment=[l]{//},
  morecomment=[n]{/*}{*/},
  morestring=[b]",
  morestring=[b]',
  morestring=[b]"""
}

\graphicspath{ {images/} }


\makeatletter
\newcommand{\thickhline}{%
    \noalign {\ifnum 0=`}\fi \hrule height 1pt
    \futurelet \reserved@a \@xhline
}
\newcolumntype{"}{@{\hskip\tabcolsep\vrule width 1pt\hskip\tabcolsep}}
\makeatother

\newcommand\Tstrut{\rule{0pt}{3.3ex}}       % "top" strut
\newcommand\Bstrut{\rule[-1.3ex]{0pt}{0pt}} % "bottom" strut
\newcommand{\TBstrut}{\Tstrut\Bstrut} % top&bottom struts

\begin{document}

%------------------
% Title Page
%------------------
\begin{titlepage}
\newcommand{\HRule}{\rule{\linewidth}{0.5mm}}
\center

\textsc{\LARGE Birkbeck College}\\[1.5cm] % Name of your university/college

\textsc{\Large Msc Computer Science Project Report}

\HRule \\[0.4cm]
{ \LARGE \bfseries Learning and Video Games: Implementing an Evolutionary Agent}\\[0.4cm] % Title of your document
\HRule \\[1.5cm]


\begin{minipage}{0.4\textwidth}
\begin{flushleft} \large
\emph{Author:}\\
Monty \textsc{West} % Your name
\end{flushleft}
\end{minipage}
~
\begin{minipage}{0.4\textwidth}
\begin{flushright} \large
\emph{Supervisor:} \\
Dr. George \textsc{Magoulas} % Supervisor's Name
\end{flushright}
\end{minipage}\\[4cm]

\emph{MSc Computer Science project report, Department of
Computer Science and Information Systems, Birkbeck College,
University of London 2015}

\vspace{5mm}

\emph{This report is substantially the result of my own work,
expressed in my own words, except where explicitly indicated in
the text. I give my permission for it to be submitted to the
JISC Plagiarism Detection Service.}

\vspace{5mm}

\emph{The report may be freely copied and distributed provided the
source is explicitly acknowledged.}

\end{titlepage}

%------------------
% Abstract
%------------------

\begin{abstract}
Artificial intelligence in video games has long shunned the use of machine learning in favour of a handcrafted approach. However, the recent rise in the use of video games as a benchmark for academic AI research has demonstrated interesting and successful learning approaches. This project follows this research and explores the viability of a game-playing learning AI. Considering previous approaches, an evolutionary agent was created for a platform game based on Super Mario Bros.

The project builds on top of software developed for the Mario AI Competition, which provides the game-engine and agent interface, as well as several other pertinent features. The basic agent was constructed first and a learning framework was built to improve it, utilising a genetic algorithm. The project followed an agile methodology, revisiting design by analysing learning capability.

The aim was to produce an agent that shows meaningful improvement during learning and demonstrated unforeseen behaviours. 

!ADD EVAL HERE!

\end{abstract}

\vspace{10mm}

\iffalse
\begin{center}
\includegraphics[scale=0.3]{mario}  %What are you doing, remove this.
\end{center}
\fi

\clearpage

%------------------
% Contents
%------------------

\tableofcontents
\clearpage

%-------------------------------------------------------------------------
% Introduction (Brief Description of Topic + Fits in Field)
%-------------------------------------------------------------------------


\section{Introduction}

Artificial intelligence (AI) is a core tenant of video games, traditionally utilised as adversaries or opponents to human players. Likewise, game playing has long been a staple of AI research. However, academic research has traditionally focused mostly on board and card games and advances in game AI and academic AI have largely remained distinct.

The first video game opponents were simple discrete algorithms, such as the computer paddle in \emph{Pong}. In the late 1970s video game AIs became more advanced, utilising search algorithms and reacting to user input. In \emph{Pacman}, the ghost displayed distinct personalities and worked together against the human player \cite{pacmanghosts}. In the mid 1990s, approaches became more `agent' based. Finite State Machines (FSMs) emerged as a dominant game AI technique, as seen in games like \emph{Half-Life} \cite{halflife}. Later, in the 2000s, Behaviour Trees gained preeminence, as seen in games such as \emph{F.E.A.R.} \cite{fear} and \emph{Halo 2} \cite{halo}. These later advances borrowed little from contemporary development in academic AI and remained localised to the gaming industry.

However, with increases in processing power and the complexity of games over the last ten years many academic techniques have been harnessed by developers. For example, Monte Carlo Tree Search techniques developed for use in Go AI research has been used in \emph{Total War: Rome II} \cite{rome}. In 2008's \emph{Left 4 Dead}, Player Modelling was used to alter play experience for different users \cite[p.~10]{playermod}. Furthermore, AI and related techniques are no longer only being used as adversaries. There has been a rise in intelligent Procedural Content Generation in games in recent years, in both a game-world sense (for example \emph{MineCraft} and \emph{Terraria}) and also a story sense (for example \emph{Skyrim's} Radiant Quest System) \cite{skyrim}.

Moreover, games have recently enjoyed more consideration in academic research. Commercial games such as \emph{Ms. Pac Man}, \emph{Starcraft}, \emph{Unreal Tournament} and \emph{Super Mario Bros.} and open-source games like \emph{TORCS} \cite{torcs} and \emph{Cellz} \cite{cellz} have been at the centre of recent competitions and papers \cite{panorama} \cite{marioaicomp}. 

Learning has long been a staple of academic research into AI and dynamic programming, especially in robotics and board games. However, it has also had success in more niche problems, such as helicopter control \cite{rlheli} and human-computer dialogue \cite{rlhci}. Similarly the agent model is a popular approach to AI problems. It is seen in commercial applications, as mentioned above, as well as academic applications, such as board game AI. Furthermore, the agent model's autonomous nature makes it particularly suited to utilising biologically inspired learning.


%-----------------------------------------------------------------------
% Definitions
%-----------------------------------------------------------------------
\subsection{Concept Definitions}
\label{subsec:concepts}

At this point it is useful to introduce some high level descriptions/definitions of some key concepts that will be used in this report.

\subsubsection{Intelligent Agents (IAs)}

\begin{figure}[t]
	\centering
	\includegraphics[scale=0.6]{intelligentagent.png}
	\caption{Illustration of an intelligent agent, taking from \cite[p.~32]{modernai1}}
	\label{fig:ia}
\end{figure}

An intelligent agent is an entity that uses \textbf{sensors} to perceive its \textbf{environment} and acts based on that perception through \textbf{actuators} or \textbf{effectors}. In software, this is often realised as an autonomous program or module that takes its perception of the \textbf{environment} as input an returns \textbf{actions} as output. Figure \ref{fig:ia} shows the basic structure of an intelligent agent. \cite[p.~34]{modernai3}


\subsubsection{Rule-based systems}

A rule-based system decides \textbf{actions} from \textbf{inputs} as prescribed by a \textbf{ruleset} or \textbf{rule base}. A \textbf{semantic reasoner} is used to manage to the relationship between input and the ruleset. This follows a \textbf{match-resolve-act} cycle, which first finds all rules matching an input, chooses one based on a conflict strategy and then uses the rule to act on the input, usually in the form of an output. \cite[pp.~28-29]{rbsys}

\subsubsection{Biologically Inspired Learning} 

Several computational learning approaches have derived their inspiration from learning in animals and humans. Two such approaches are relevant to this project: Reinforcement learning, strategy modelled on simplistic interpretation of how animal learn behaviour from their environment \cite[s.~1.2]{suttonrl}; and evolutionary computation, algorithms that apply Darwinian principles of evolution \cite{ev-comp}.


\subsubsection*{Reinforcement Learning}

A reinforcement learning agent focuses on a learning problem, with its goal to maximise \textbf{reward}. Given a current \textbf{state} the agent chooses an \textbf{action} available to it, which is determined by a \textbf{policy}. This action maps the current \textbf{state} to a new \textbf{state}. This \textbf{transition} is then evaluated for its \textbf{reward} . This \textbf{reward} often affects the \textbf{policy} of future iterations, but \textbf{policies} may be stochastic to some level. \cite[s.~1.3]{suttonrl}

\subsubsection*{Evolutionary Computation}

Evolutionary computation encompasses many learning algorithms, two of which are described in detail below. The process looks to optimise a data representation of the problem through random variation and selection in a \textbf{population}. It commonly employs techniques similar to survival, breeding and mutation found in biological evolutionary theory.  \cite{ev-comp}
	
\subsubsection*{Genetic Algorithms (GAs)}

Genetic Algorithms are a subset of evolutionary computation. They model the solution as a \textbf{population} of \textbf{individuals}. Each \textbf{individual} has a set of \textbf{genes} (a  \textbf{genome}), which can be thought of as simple pieces of analogous information (most often in the form of bit strings). Each \textbf{individual} is assessed by some \textbf{fitness function}. This assessment can be used to cull the \textbf{population}, akin to survival of the fittest, or to increase the individual's chance of influencing the next \textbf{population}. The new \textbf{population} is created by \textbf{breeding}, using a combination of the following: \textbf{crossover} of the \textbf{genome} from two (or more) \textbf{individuals} (akin to sexual reproduction), \textbf{mutation} of the \textbf{genes} of one \textbf{individual} (akin to asexual reproduction) and \textbf{re-ordering} of the \textbf{genes} of one \textbf{individual}. Each new \textbf{population} is called a \textbf{generation}. \cite[p.~7]{mitchellga}

\subsubsection*{Evolution Strategies (ESes)}

Evolution Strategies are another example of evolutionary computation. They differ from standard Genetic Algorithms by using \textbf{truncation selection} before breeding. The top $\mu$ individuals of the population are chosen (usually by fitness) and bred to create $\lambda$ children. ES notation has the following form:  $\pmb{(\mu / \rho \  \overset{+}{,} \  \lambda)}$. $\rho$ denotes the number of individuals from $\mu$ used in the creation of a single $\lambda$, (i.e. number of parents of each child) this report will only consider the case $\rho = 1$. The $+$ and $,$ are explained below:  \cite[p.~6-10]{es-book} \cite[s.~4.1.2]{ecj-manual}
\begin{description}
	\item[$\pmb{(\mu \  , \   \lambda)}$]  Denotes an ES that has a population size of lambda. The top $\mu$ individuals are taken from the $\lambda$ in generation $g-1$, which then produce $\lambda$ children for generation $g$. This is done by creating $\lambda / \mu$ clones of each $\mu$ and then mutating them individually. 
	\item[$\pmb{(\mu  + \lambda)}$] Differs from the $\pmb{(\mu \  , \   \lambda)}$ variant by adding the $\mu$ individuals chosen from generation $g-1$ to the new generation $g$ after the mutation phase. Hence the population size is $\lambda + \mu$.
\end{description}

\subsubsection{Online/Offline Learning}
\begin{description}
	\item[Offline] An offline (or batch) learner trains on an entire dataset before applying changes. 
	\item[Online] A online learner reacts/learns from data immediately after each datapoint.
\end{description} 


%-----------------------------------------------------------------------
% Motivation
%-----------------------------------------------------------------------
\subsection{Motivation}

Ventures in utilising learning in commercial video games have been limited and largely ineffectual. Game development works on strict cycles and there are limited resources to invest into AI research, especially if the outcome is uncertain. Furthermore, one player playing one game produces a very small data set, making learning from the player challenging. \cite{evolutioningamedesign}

However, there are many reasons why good execution of these techniques is desirable, especially biologically inspired learning. Humans must learn and react to environments and scenarios during games, based purely on their perception of the game (and not its inner working). Having non-playable characters do the same may produce a more believable, immersive and relatable AI. Secondly, such learning algorithms produce agents that can respond well in new situations (over say FSMs or discrete logic), hence making new content easy to produce or generate. Lastly, modern games have large and diverse player bases, having a game that can respond and personalise to a specific player can help cater to all. \cite[p.~7, p.~13]{panorama}

Despite the lack of commercial success, video games can act as great benchmark for learning agents. They are designed to challenge humans, and therefore will challenge learning methods, especially those inspired by biological processes. Games naturally have some level of learning curve associated with playing them (as a human). Also, games require quick reactions to stimulus, something not true of traditional AI challenges such as board games. Most games have some notion of scoring suitable for a fitness function. Lastly, they are generally accessible to students, academics and the general public alike. \cite[p.~9]{panorama} \cite[p.~1]{marioaicomp} \cite[p.~2]{2012the}

As such, games have now been utilised in several academic AI competitions. These competitions are the forefront of research and development into learning techniques in video games, and will be explored in more detail in section \ref{subsec:gameaicomps}.

\vspace{\baselineskip}

Exploring the use of learning techniques for use in video games is a challenging and eminent area of research, with interest from both the video game and computation intelligence communities. This project is motivated by this fact and influenced by the variety of previous approaches taken in these competitions and the unexpected results they produced. 


%-----------------------------------------------------------------------
% Aims and Objectives
%-----------------------------------------------------------------------
\subsection{Aim and Objectives}

The aim of the project is to explore the use of behavioural learning techniques in creating a game-playing agent-based AI. This will be achieved by producing an intelligent agent, developed by a evolutionary algorithm, that plays a 2D side-scrolling platform game.

\subsubsection*{Objectives}

\begin{enumerate}

	\item \label{obj:design}
	\textbf{Design} \\
	Influenced by previous approaches, design an agent and learning procedure that can demonstrate meaningful improvement during learning.
	
	\item \label{obj:impl}
	\textbf{Implementation} \\
	Implement agent and learning designs in a modular, customisable and test-driven manner, utilising external libraries and features of the game software.
	
	\item \label{obj:test}
	\textbf{Testing} \\
	Provide significant test coverage to the functionality of the implementation.
	
	\item \label{obj:learn}
	\textbf{Learning} \\
	Grant the agent a large and diverse testbed from which to learn as well as ample time and resources to do so.
	
	\item \label{obj:eval}
	\textbf{Evaluation} \\
	Evaluate both the competence and learning capability of the agent and compare to alternative approaches.

\end{enumerate}

%-----------------------------------------------------------------------
% Methodology
%-----------------------------------------------------------------------
\subsection{Methodology}

The two central components to this project, the agent and the learning process, will be designed, implemented and tested as two separate modules. This presents a clean separation of concerns. The agent module will be completed first, followed by the learning module. A third module will also be created, which will allow the agent to play generated levels and receive a score.

It is foreseeable that certain parts of the project will be exploratory and as such a redesign of the either the agent or the learning modules may become required. Hence it will be important to take an agile and iterative approach to production, revisiting past objectives if need be.

The open-source Mario benchmark software\footnote{Available at http://www.marioai.org/gameplay-track/getting-started} (used in the Mario AI Competition) will form the initial codebase for this project. It is written in Java and contains the game-engine and useful features such as level generation and level playing classes.

I intend to use Scala as the main language of this project as I believe that its functional and concurrent features will suit the problem. Scala and Java are widely compatible (as they both compile JVM bytecode) so this should not require a significant amount of work. However, the codebase will be assessed for the suitability of this approach, and Java 8 will be used if Scala proves to be counterproductive.

Furthermore the software will be assessed for the inclusion of build tools, such as \emph{sbt} and \emph{Maven}; testing frameworks, such as \emph{ScalaMock} and \emph{Mockito} and logging, such as \emph{log4j} and Scala's inbuilt logging package.

\subsubsection*{Objective \ref{obj:design}: Design}
\label{meth:design}

The design of both the agent and learning module will follow the approach adopted in previous work regarding the REALM agent, which is discussed in detail in section \ref{ssec:realm}.

As in previous work, the agent module developed for the project will be a rule-based system that maps environment based conditions to explicit key presses. However, the proposed approach will differ from the REALM agent by exploring several additional perceptions of the available sensory information. These perceptions will be limited to `visible' environment (e.g. the direction mario is moving, presence of enemies etc.). They will be measured and distilled into a collection of simple observations.

The learning module will utilise an offline genetic algorithm as seen in previous work, which will be described later in section \ref{ssec:realm}. Agents will be individuals, with rulesets as genomes. Fitness of an individual will be determined by playing generated levels. However, different approaches to mutation, crossover and reordering will be explored, as well as careful consideration of calculating fitness from level playing statistics.

\subsubsection*{Objective \ref{obj:impl}: Implementation}
\label{meth:impl}

The agent module will conform to the \emph{Agent} interface included in the benchmark. This will allow the agent to used in other areas of the benchmark software, such as the game-playing and evaluation classes. Each agent will be defined by its ruleset, the contents of which will determine its behaviour. The agent will implement a semantic reasoner for the ruleset, returning a action as prescribed by the chosen rule.

The agent will gather its sensory information from the game's \emph{Environment} class, which reports visible properties of the level scene. If necessary this class will be extended to include any missing `visible' data. As discussed in \ref{meth:design}, this will be simplified to a list of observations and compared against the agent's ruleset.

\vspace{\baselineskip}

The learning module will utilise an external library alongside the level playing module. Many evolutionary computation libraries exist for Java (and therefore Scala), \emph{ECJ} \cite{ecj} and \emph{JGAP} \cite{jgap} will both be considered for use in this project.

The implementation will manage the integration of this library with both the agent and level playing modules. The algorithm will evolve a simple data structure representation of rulesets, inject them into agents and assess those agents by having them play several carefully parametrised levels. The statistics returned from these levels will form the base variables of the fitness function, with multipliers being configurable. To aid in improving the learning process, these parameters will be held externally and fitness values will be logged.

\subsubsection*{Objective \ref{obj:test}: Testing}
\label{meth:test}

Test coverage the agent module will handled by black-box testing. Unit tests will be written to test the semantic reasoner and the environment observation methods.

Due to the stochastic nature of genetic algorithms, testing of the learning module will be limited. However, the breeding can be tested by verifying that children stay within parameters. Some evolutionary libraries, such as JGAP, provide several unit tests. If available these will become the primary source of test coverage for the module.

\subsubsection*{Objective \ref{obj:learn}: Learning}
\label{meth:learn}

Optimising the learning parameters, including the parameters of the breeding phase, fitness function and level playing, will be an important stage of the project. Assessment of the parameters used in previous agent such as REALM (discussed in \ref{ssec:realm}) D. Perez et al. \cite{gramev} and Agent Smith \cite{agentsmith} will inform the initial values. From there learning module logs will be analysed for improvements. For example, if there is a lot of variation in fitness then perhaps mutation should be restricted, or if the average fitness does not eventually level out then the further generations should be created.

The design of the agent can also influence the effectiveness of the learning algorithm. The size of the search space is determined by the conditions and actions of the rulesets, the reduction of which could improve evolutionary capability. Hence, the learning phase of the project may inform a redesign of the agent, which is one of the main reasons this project will take an agile approach.

The learning itself is likely to be a time consuming, computationally heavy procedure. To assist in providing ample resources to this process the project will have access to two 8 core servers, as well as a laptop with an Intel i7 processor.

\subsubsection*{Objective \ref{obj:eval}: Evaluation}
\label{meth:eval}

On the conclusion of the learning process the best/final agent will be extracted and evaluated. This will be done by using the level playing module. The agent will go through an extensive set of levels, based on the approach taken by the 2010 Mario AI Competition.

The primary comparison will be with a handcrafted ruleset, which will assess the significance of the agent evolution. Other comparisons can be drawn against agents that are included as examples in the benchmark, such as the \emph{ForwardJumpingAgent} that was used for similar comparisons in the 2009 competition, as well as other entrants into the competition \cite[p.~7]{2010the}.

The second part of this objective is to evaluate the learning procedure. Figures such as average and maximum generation fitness can provide an insight into the effectiveness of the genetic algorithm. Furthermore, a baseline for these values can be provided by having the handcrafted agent play the levels alongside the evolving agent. The final evaluation report will provide an analysis of these figures.

%-----------------------------------------------------------------------
% Report Structure
%-----------------------------------------------------------------------
\subsection{Report Structure}

This report will cover previous approaches to the project's aim; the design, implementation of the agent and learning process; and evaluate both the results and the project as a whole.

Section 2 details existing work, specifically relevant entrants to Game AI competitions. It will consider their approaches to both learning and agent design for both effectiveness and significance to learning in video games. It will look in particular at the Mario AI Competition and the winner entrant in 2010, the REALM agent.

Section 3 will cover the projects specification, discussing functional and non-functional requirements. It will also include a description of the major dependencies that influenced project design.

Section 4 will explain the design, implementation and testing of the rule-based agent. It will demonstrate how the agent was built to allow it be evolved by a genetic algorithm, as well as how it perceives its environment and choosing an action. Reasons for the choice of project language and build tools is also included here. 

Section 5 will detail the development of the level playing module.This contains an account of the modifications that had to be made to the game engine software. It covers the extension to the game engine and how it as designed and implemented with a view to parametrisation and level variety.

Section 6 explains the choice of genetic algorithm and the basic parameters used. It also describes extensions made to the learning library to allow it to effectively evolve the agent in an easily customisable and observable fashion. Lastly, it details how specific mutataion and fitness parameters were chosen in response to initial learning runs in order to improve the process.

Section 7 evaluates the agent framework for its capability by considering 3 hand crafting agent rulesets. It then compares these agents to the final learnt agent, assessing it for attainment, as well as interesting and unexpected behaviour. It also studies the effectiveness of the learning algorithm by examining metrics such as fitness increase over generations. Finally, it concludes with an analysis of the project as a whole, with a discussion on extensions and improvements to the learning and agent modules and the overall methodology.

\clearpage

%___________________________________________
%*************************************************************
% Existing Work
%___________________________________________
%^^^^^^^^^^^^^^^^^^^^^^^^^^^^^^^^^^^^^^^^^^^^^^^^^^^

\section{Existing Work}

%-----------------------------------------------------------------------
% Learning in Games (Current Work in Field)
%-----------------------------------------------------------------------

\subsection{Learning Agents and Game AI Competitions}
\label{subsec:gameaicomps}

Over the last few years several game based AI competitions have been run, over a variety of genres. These competitions challenge entrants to implement an agent that plays a game and is rated according to the competitions specification. They have attracted both academic \cite[p.~2]{2012the} and media interest \cite[p.~2]{marioaicomp}. The competition tend to encourage the use of learning techniques. Hence, several interesting papers concerning the application of biologically inspired learning agents in video games have recently been published. Approaches tend to vary widely, modelling and tackling the problem very differently and specialising techniques in previously unseen ways. \cite[p.~11]{2012the}

Some brief details of the competitions which are of relevance to this project are compiled in to Table \ref{tab:gameaicomp}. The Mario AI Competition is also explored in more detail below.

\begin{table}
  \begin{adjustwidth}{-4cm}{-4cm}
  \begin{center} \small
    \begin{tabular}{ | >{\raggedright}p{2cm} | >{\raggedright}p{2cm} | p{7.5cm} |}
    \hline
    \textbf{Genre} & \textbf{Game} & \textbf{Description} \TBstrut \\ \thickhline
    Racing & TORCS (Open-source) \cite{torcs} &  
    \textbf{The Simulated Car Racing Competition}\newline Competitors enter agent drivers, that undergo races against other entrants which include qualifying and multi-car racing. The competition encourages the use of learning techniques. \cite{scrc}
      \\ \hline
    First Person Shooter (FPS) & Unreal Tournament 2004 &
    \textbf{The 2K BotPrize}\newline Competitors enter `bots' that play a multi-player game against a mix of other bots and humans. Entrants are judged on Turing test basis, where a panel of judges attempt to identify the human players. \cite{2kbot}
       \\ \hline
    Real Time Strategy (RTS) & Starcraft &
    \textbf{The Starcraft AI Competition}\newline Agents play against each other in a 1 on 1 knockout style tournament. Implementing an agent involves solving both micro objectives, such as path-planning, and macro objectives, such as base progression. \cite{starcomp}
       \\ \hline
    Platformer & Infinite Mario Bros (Open-source) &
    \textbf{The Mario AI Competition}\newline Competitors submit agents that attempt to play (as a human would) or create levels. The competition is split into `tracks', including Gameplay, Learning, Turing and Level Generation. In Gameplay, each agent must play unseen levels, earning a score, which is compared to other entrants. \cite{2012the} 
       \\ \hline
    
    \end{tabular}
  \end{center}
  \end{adjustwidth}
  \caption{\small This table summarises some recent game AI competitions \cite{2014how}}
  \label{tab:gameaicomp}
\end{table}

\subsubsection{The Mario AI Competition}
\label{subsec:marioaibench}

The Mario AI Competition, organised by Sergey Karakovskiy and Julian Togelius, ran between 2009-2012 and used an adapted version of the open-source game Infinite Mario Bros. From 2010 onwards the competition was split into four distinct `tracks'. We shall focus on the unseen Gameplay track, where agents play several unseen levels as Mario with the aim to finish the level (and score highly). \cite{marioaicomp} \cite{2012the}

\subsubsection*{\hspace{6pt}Infinite Mario Bros.}

Infinite Mario Bros (IMB) \cite{imb} is an open-source clone of Super Mario Bros.~3, created by Markus Persson. The core gameplay is described as a \emph{Platformer}. The game is viewed side-on with a 2D perspective. Players control Mario and travel left to right in an attempt to reach the end of the level (and maximise score). The screen shows a short section of the level, with Mario centred. Mario must navigating terrain and avoid enemies and pits. To do this Mario can move left and right, jump, duck and speed up. Mario also exists in 3 different states, \emph{small}, \emph{big} and \emph{fire} (the latter of which enables Mario to shoot fireballs), accessed by finding powerups. Touching an enemy (in most cases) reverts Mario to a previous state. Mario dies if he touches an enemy in the \emph{small} state or falls into a pit, at which point the level ends. Score is affected by how many coins Mario has collected, how many enemies he has killed (by jumping on them or by using fireballs or shells) and how quickly he has completed the level. \cite[p.~3]{2012the}

\subsubsection*{\hspace{6pt}Suitability to Learning}

The competitions adaptation of IMB (known henceforth as the `benchmark') incorporates a tuneable level generator and allows for the game to be sped-up by removing the reliance on the GUI and system clock. This makes it a great testbed for reinforcement learning. The ability to learn from large sets of diverse data makes learning a much more effective technique. \cite[p.~3]{2012the}

Besides that, the Mario benchmark presents an interesting challenge for learning algorithms. Despite only a limited view of the ``world" at any one time the state and observable space is still of quite high-dimension. Though not to the same extent, so too is the action space. Any combination of five key presses per timestep gives a action space of $2^5$ \cite[p.~3]{2012the}. Hence part of the problem when implementing a learning algorithm for the Mario benchmark is reducing these search spaces. This has the topic of papers by Handa and Ross and Bagnell \cite{rossbagnell} separately addressed this issue in their papers \cite{handa} and \cite{rossbagnell} respectively.

Lastly, there is a considerable learning curve associated with Mario. The simplest levels could easily be solved by agents hard coded to jump when they reach an obstruction, whereas difficult levels require complex and varied behaviour. For example, traversing a series of pits may require a well placed series of jumps, or passing a group of enemies may require careful timing. Furthermore, considerations such as score, or the need to backtrack from a dead-end greatly increase the complexity of the problem. \cite[p.~3, p.~12]{2012the}


%-----------------------------------------------------------------------
% Previous Agents
%-----------------------------------------------------------------------

\subsection{Previous Learning Agent Approaches}
\label{ssec:prevagents}

Agent-based AI approaches in commercial games tend to focus on finite state machines, behaviour trees and rulesets, with no learning component. Learning agents are more prevalent in AI competitions and academia, where it is not only encouraged, but viewed as an interesting research topic \cite[p.~1]{marioaicomp}. Examples from both standpoints are compiled in Table \ref{tab:agents}.

\subsubsection{Evolutionary Algorithms}

In section \ref{subsec:concepts} we presented some basic concepts of evolutionary and genetic computing. This approach is a common choice of learning methods used in game-playing agents. D. Perez et al. note in their paper \cite[p.~1]{gramev} that evolutionary algorithms are particularly suitable for video game environments: 
\begin{quote}\itshape
`Their stochastic nature, along with tunable high- or low-level representations, contribute to the discovery of non-obvious solutions, while their population-based nature can contribute to adaptability, particularly in dynamic environments.'
\end{quote}
The evolutionary approach has been used across several genres of video games. For example, \emph{neuroevolution}, a technique that evolves neural networks, was used in both a racing game agent (by L. Cardamone \cite[p.~137]{scrc}) and a FPS agent (by the UT$\wedge$2 team \cite{2kbot}). Perhaps the most popular approach was to use genetic algorithms (GAs) to evolve a more traditional game AI agent. R. Small used a GA to evolve a ruleset for a FPS agent \cite{agentsmith}, T. Sandberg evolved parameters of a potential field in his Starcraft agent \cite{emapf} , \emph{City Conquest's} in-game AI used an agent-based GA-evolved build plan \cite{evolutioningamedesign} and D. Perez et al. used a grammatical evolution (a GA variant) to produce behaviour trees for a Mario AI Competition entry \cite{gramev}.

\subsubsection{Multi-tiered Approaches}

Several of the most successful learning agents take a multi-tiered approach. By splitting high-level behaviour from low-level actions agents can demonstrate more a complex, and even human-like, performance. For example, COBOSTAR, an entrant in the 2009 Simulated Car Racing Competition, used offline learning to determine high-level parameters such as desired speed and angle alongside a low-level crash avoidance module \cite[p.~136]{scrc}. UT$\wedge$2 used learning to give their FPS bot broad human behaviours and a separate constraint system to limit aiming ability \cite{2kbot}.  Overmind, the winner of the 2010 Starcraft Competition, planned resource use and technology progression at a macro level, but used A* search micro-controllers to coordinate units \cite{overmind}. 

\vspace{\baselineskip} \noindent
One learning agent that successfully utilised both an evolutionary algorithm and a multi-tiered approach is the Mario agent REALM, which is explored in more detail below. 

\subsubsection{REALM}
\label{ssec:realm}

The REALM agent, developed by Slawomir Bojarski and Clare Bates Congdon, was the winner of the 2010 Mario AI competition, in both the unseen and learning Gameplay tracks. REALM stands for \textbf{R}ule Based \textbf{E}volutionary Computation \textbf{A}gent that \textbf{L}earns to Play \textbf{M}ario. REALM went through two versions (V1 and V2), with the second being the agent submitted to the 2010 competition.

\subsubsection*{\hspace{6pt}Rule-based}

Each time step REALM creates a list of binary observations of the current scene, for example {\footnotesize IS\_ENEMY\_CLOSE\_LOWER\_RIGHT} and {\footnotesize IS\_PIT\_AHEAD}. Conditions on observations are mapped to actions in a simple ruleset. These conditions are ternary (either {\footnotesize TRUE}, {\footnotesize FALSE} or {\footnotesize DONT\_CARE}) \cite[p.~85]{realm}. A rule is chosen that best fits the current observations, with ties being settled by rule order, and an action is returned \cite[p.~86]{realm}.

Actions in V1 are explicit key-press combinations, whereas in V2 they are high-level plans. These plans are passed to a simulator, which reassesses the environment and uses A* to produce the key-press combination. This two-tier approach was designed in part to reduce the search space of the learning algorithm. \cite[pp.~85-87]{realm}

\subsubsection*{\hspace{6pt}Learning}

REALM evolves ruleset using an ES for 1000 generations. The best performing rule set from the final generation was chosen to act as the agent for the competition. Hence, REALM is an agent focused on offline learning. \cite[pp.~87-89]{realm}

Populations have a fixed size of 50 individuals, with each individual's genome being a ruleset. Each rule represents a gene and each individual has 20. Initially rules are randomised, with each condition having a 30\%, 30\%, 40\% chance to be {\footnotesize TRUE}, {\footnotesize FALSE} or {\footnotesize DONT\_CARE} respectively.

Individuals are evaluated by running through 12 different levels. The fitness of an individual is a modified score, averaged over the levels. Score focuses on distance, completion of level, Mario's state at the end and number of kills. Each level an individual plays increases in difficulty. Levels are predictably generated, with the seed being recalculated at the start of each generation. This is to avoid over-fitting and to encourage more general rules.

REALM used the $\pmb{(\mu  + \lambda)}$ variant ES, with $\mu = 5$ and $\lambda = 45$ (i.e. the best 5 individuals are chosen and produce 9 clones each). Offspring are exposed to: \textbf{Mutation}, where rule conditions and actions may change value; \textbf{Crossover}, where a rule from one child may be swapped with a rule from another child\footnote{This is similar to a  $\pmb{(\mu/\rho  + \lambda)}$ ES approach with $\rho = 2$, but crossover occurs in the mutation phase and between all children, rather than specifically with children from another parent.} and \textbf{Reordering}, where rules are randomly reordered. These occur with probabilities of 10\%, 10\% and 20\% respectively \cite[p.~88]{realm}. Unfortunately, the method employed to perform these operations is not clearly explained in the REALM paper \cite{realm}.

\subsubsection*{\hspace{6pt}Performance}

The REALM V1 agent saw a larger improvement over the evolution, but only achieved 65\% of the V2 agent's score on average. It is noted that V1 struggled with high concentrations of enemies and large pits. The creators also assert that the V2 agent was more interesting to watch, exhibiting more advanced and human-like behaviours. \cite[pp.~89-90]{realm}

The ruleset developed from REALM V2 was entered into the 2010 unseen Gameplay track. It not only scored the highest overall score, but also highest number of kills and was never disqualified (by getting stuck in a dead-end). Competition organisers note that REALM dealt with difficult levels better than other entrants. \cite[p.~10]{2012the}


\begin{table}
  \begin{adjustwidth}{-4cm}{-4cm}
  \begin{center} \small
    \begin{tabular}{ | >{\raggedright}p{2.2cm} | >{\raggedright}p{3.5cm} | p{6.8cm} |}
    \hline
    \textbf{Name} & \textbf{Game/Competition} & \textbf{Approach} \TBstrut \\ \thickhline
    
    M. Erickson \cite{2012the} & 2009 Mario AI Competition &
    A crossover heavy GA to evolve an expression tree.
    \\ \hline
    E. Speed \cite{2010the} & 2009 Mario AI Competition &
    GA to evolve grid-based rulesets. Ran out of memory during the competition.
    \\ \hline
    S. Polikarpov \cite[p.~7]{2010the} & 2009-10 The Mario AI Competition &
    Ontogenetic reinforcement learning to train a neural network with action sequences as neurons.
    \\ \hline
    REALM \cite{realm} & 2010 Mario AI Competition &
    GA to evolve rulesets mapping environment to high-level behaviour.
    \\ \hline
    D. Perez et al \cite{gramev} & 2010 Mario AI Competition &
    Grammatical evolution with a GA to develop behaviour trees. 
    \\ \hline
    FEETSIES \cite{feetsies} & 2010 Mario AI Competition &
    ``Cuckoo Search via L\'evy Flights" to develop a ruleset mapping an observation grid to actions. 
    \\ \hline
    COBOSTAR \cite[p.~136]{scrc} & 2009 Simulated Car Racing Competition &
    Covariance matrix adaptation evolution strategy to map sensory information to target angle and speed.
    Online reinforcement learning to avoid repeating mistakes.
    \\ \hline
    L. Cardamone \cite[p.~137]{scrc} & 2009 Simulated Car Racing Competition &
    Neuroevolution to develop basic driving behaviour.
    \\ \hline
    Agent Smith \cite{agentsmith} & Unreal Tournament 3 &
    GAs to evolve very simple rulesets, which determine basic bot behaviour.
     \\ \hline
    UT$\wedge$2 \cite{2kbot} & 2013 2K Botprize &
    Neuroevolution with a fitness function focused on being `human-like'.
    \\ \hline
    T. Sandberg \cite{emapf} & Starcraft &
    Evolutionary algorithms to tune potential field parameters.
     \\ \hline
    Berkeley Overmind \cite{overmind} & The Starcraft AI Competition &
    Reinforcement learning to tune parameters for potential fields and A* search.
    \\ \hline
    In-game Opponent AI \cite{evolutioningamedesign} & City Conquest &
    GAs to evolve build plans with fitness measured in a 1-on-1 AI match.
    \\ \hline
    In-game Creature AI\cite{blackandwhite} & Black \& White &
    Reinforcement Learning applied to a neural network representing the creatures desires.
    \\ \hline
    In-game Car AI \cite{projectgothamracing} & Project Gotham Racing &
    Reinforcement learning to optomise racing lines.
     \\ \hline
    
    \end{tabular}
  \end{center}
  \end{adjustwidth}
  \caption{\small Biologically inspired learning, agent-based approaches to game playing AI}
  \label{tab:agents}
\end{table}



%___________________________________________
%*************************************************************
% Project Specification
%___________________________________________
%^^^^^^^^^^^^^^^^^^^^^^^^^^^^^^^^^^^^^^^^^^^^^^^^^^^

\section{Project Specification}

\subsection{Functional Requirements}
Functionally, the project can be split into three parts: the agent framework, which is responsible for gathering sensory information from the game and producing an action; level generation and playing, which is responsible for having an agent play generated levels with varying parameters; and the learning module, which will apply a genetic algorithm to a representation of an agent, with an aim to improving its level playing ability. 

\subsubsection{Agent}
The agent framework will implement the interface supplied in the Mario AI benchmark; receive the game \emph{Environment} and produce an \emph{Action}. It must be able to encode individual agents into a simple format (e.g. a bit string or collection of numbers). Additionally, it should be able to encode and decode agents to and from external files. 

The framework will be assessed in three ways. Firstly on its complexity, keeping the search space of the encoded agent small is important for the learning process. Secondly on its speed, the agent must be able to respond within one game tick. Thirdly on its capability, the framework must facilitate agents that can complete the easiest levels, and attempt the hardest ones. It is also required that human created agent encoding(s) be written (within the framework) to assist this assessment.

\subsubsection{Level Playing}
The level generation and playing module must be able to generate and play levels using an agent, producing a score on completion. It should extend the existing framework included in the Mario AI Benchmark software. Furthermore, it should be able to read parameters for generation and scoring from an external file.

The module will be evaluated on the level of variety in generated levels and the amount of information it can gather in order to score the agent.

\subsubsection{Learning}
The learning module should utilise a genetic algorithm to evolve an agent (in encoded form). It should also ensure that as many as possible of the parameters that govern the process can be held in external files, this included overall strategy as well fine grained detail (e.g. mutation probabilities and evaluation multipliers). Where impossible or inappropriate to hold such parameters externally it must be able to read them dynamically from their governing packages, for example boundaries on the agent encoding should be loaded from the agent package, rather than held statically in the learning module. It must have the facility to report statistics from learning runs, as well as write out final evolved agents.

The learning module will also be assessed on three counts. Firstly, learning should not run for too long, grating the freedom to increase generation count or adjust parameters. Secondly, the learning process should demonstrate a meaning improvement to the agent over generations as this demonstrates an effective genetic algorithm. Thirdly, the final evolved agent will be assessed, using the level playing package. It will tested against the human created agents and analysed for behaviours and strategies not considered during their creation. 

\subsection{Non-functional requirements}

Both the level playing module and the agent framework should not prevent or harm thread safety, allowing multi-threading in the learning module. Each part should be deterministic, i.e. if given the same parameter files will always produce the same results. Lastly, the entire project must have the ability to be packaged and run externally.

\subsection{Major Dependencies}
The project has two major dependencies: a game engine and a learning library. Their selection influenced the design of all aspects of the project and hence are included here. 

As previously mentioned, the project will extend the Mario AI Benchmark. As previously discussed in Section \ref{subsec:marioaibench} the Mario AI Benchmark is an open-source Java code-base, build around a clone of the game Super Mario Bros. 3. It was chosen for its aforementioned suitability to learning and for its other pertinent features, including an agent interface, level playing and generation package and other useful additions (e.g. the ability to turn off the reliance on the system clock).

The use of the Mario AI Benchmark restricts the choice of language (discussed further in Section \ref{subsec:langchoice}) to those that can run on the JVM. Hence, ECJ was chosen as the learning library as it is written in Java and is available as both source-code and packaged jar. Furthermore, ECJ fit the project specification very well: it provides support for GAs and ESes, its classes and settings are decided at runtime from external parameter files, it is highly flexible and open to extension, and has facilities for logging run statistics.

Although the intention was to include these dependencies as packaged jars, modification of their source code was necessary (which was available on academic licences). This code was modified in Java, packaged and included as dependencies in the main project. Details of the modifications can found in sections \ref{subsec:enginemod} and \ref{subsec:ecjmod}.

%___________________________________________
%*************************************************************
% Agent
%___________________________________________
%^^^^^^^^^^^^^^^^^^^^^^^^^^^^^^^^^^^^^^^^^^^^^^^^^^^

\section{Agent Framework}

%-----------------------------------------------------------------------
% Design
%-----------------------------------------------------------------------

\subsection{Design}

\begin{figure}[t]
	\centering
	\includegraphics[scale=0.5]{diagrams/AgentClassDiagram.png}
	\caption{UML class diagram of the agent framework.}
	\label{fig:aumlcd}
\end{figure}

The benchmark offers an \emph{Agent} interface, which when extended can be passed into the game playing portion of the software. Every game tick the current \emph{Environment} is passed to the agent, and then the agent is asked for an \emph{Action}, which is a set of explicit key press for Mario to perform.

The agent framework's basis will be extending this interface. Similar to REALM's agent, it will be a rule-based system. A UML class diagram of the system is given in Figure \ref{fig:aumlcd}.

\vspace{\baselineskip}

Each \emph{Agent} instance is initialised with a ruleset, consisting of a list of rules, each containing an \emph{Action}. On receiving the current game \emph{Environment} it is passed to a number of perception objects. Each one is responsible for a exactly one measurement (e.g. Enemy ahead, Mario moving to right etc.). These measurements are collected in to an observation object and stored in the agent.

When asked for an action, the \emph{Agent} passes this observation to its ruleset, which tests it against each rule. Rules contain conditions on what perception measurements should be, which determine its score for a given observation. The highest scoring  rule (with a conflict strategy for ties) is then asked for its action, which is returned from the agent.

\vspace{\baselineskip}

Perceptions are hard coded and as such all of an agents behaviour is determined by its ruleset. If ruleset are implemented in a persistable fashion it ensures that agents can be read and written to external files. Furthermore, rulesets are immutable and only information received during the current game tick is considered. Hence, agents have no state and are deterministic and thread safe (assuming the game engine is).

The capability of this framework is closely tied to the choice of perceptions, which therefore must be chosen with careful consideration. The perceptions chosen were based upon those used in the REALM v1 agent \cite[p.~85]{realm}. The following is the final list of perceptions chosen:
\begin{description}
\item[MarioMode] Measures whether Mario is \textbf{small}, \textbf{big} or \textbf{fire} (capable of throwing fireballs).
\item[JumpAvailable] Detects if Mario has the ability to jump (i.e. on the ground or gripping a wall with the jump key off).
\item[OnGround] Measures whether Mario is in the air or not.
\item[EnemyLeft] Detects if an enemy is to the left of Mario.
\item[EnemyUpperRight] Detects if an enemy is to the upper right of Mario.
\item[EnemyLowerRight] Detects if an enemy is to the lower right or directly to the right of Mario.
\item[ObstacleAhead] Detects if there is a block or terrain directly to the right of Mario.
\item[PitAhead] Detects if there is a pit (falling into pits will end the level) to the right of Mario.
\item[PitBelow] Detects if there is a pit directly below Mario.
\item[MovingX] Measures the direction Mario is moving horizontally.
\item[MovingY] Measures the direction Mario is moving vertically.
\end{description}

The set of actions Mario can take is any combination of 6 key presses: \textbf{left}, \textbf{right}, \textbf{up}, \textbf{down}, \textbf{jump} and \textbf{speed} (which also shoots a fireball if possible). It was decided that this set would be restricted for use in the ruleset agent, only allowing cobinations of \textbf{left}, \textbf{right}, \textbf{jump} and \textbf{speed}. This was to reduce the size of the rules, which reduces the search space of any learning process. The \textbf{up} key performed no function and the \textbf{down} key made Mario crouch, which seemed too niche to be useful during the design process.


%-----------------------------------------------------------------------
% Lang + Tools
%-----------------------------------------------------------------------

\subsection{Language and Tools}

\subsubsection{Scala}
\label{subsec:langchoice}
The dependency on the Mario AI Benchmark restricts the language choice to those that can run on the JVM. Hence, Scala was chosen to be the primary language of the project. Its functional programming elements (pattern matching, list processing etc.) are very applicable to the ruleset and its semantic reasoner. Scala's focus on immutability aids in maintaining the thread safety requirement. Furthermore, ECJ's structure necessitates the use of type casting, which Scala handles elegantly. Several other Scala features were used throughout the project, such as lambdas, singletons, type aliases and case classes. 

\subsubsection{Maven}
With two major and several minor dependencies, their management is important to the project. Maven was chosen to maintain this, as well as package the main project and both the sub-projects. It was chosen for its ability to package both Java and Scala projects, keeping the tools consistent across the entire code base.


%-----------------------------------------------------------------------
% Impl
%-----------------------------------------------------------------------

\subsection{Implementation}

The design of the agent framework allows for agents to be fully determined by their ruleset. In order for an agent to be used in a evolutionary algorithm it must be represented by a simple data structure. Hence, implementation must allow for agent's rulesets to be unambiguously represented as, for example, a one dimensional collection. This sections will describe the implementation steps taken to represent rulesets as a array of 8-bit integers, utilising Scala's built in types Vector and Byte.

\subsubsection{Perceptions}

The Perception interface was implemented as an abstract class, with an index integer field and an apply method. \emph{apply} takes the game Environment and returns a Byte (an 8-bit integer) instance, representing the measurement. Perception was extended into two further abstract classes: BoolPerception, which enforces apply's return be either 1 or 0, representing true and false respectively; and BytePerception, which has a limit field and enforces apply's return is between 0 and limit (inclusively).

Concrete perceptions were implemented as objects (singletons) using the \emph{case} keyword, which allows for exhaustive pattern matching on the Perception class, as well as type safe extension by adding further case objects. Each one declares a unique index integer to the Perception superclass (starting at 0 and increasing by one for each). They implement the apply method to return their specific measurement. Figure \ref{fig:percumlcd} shows a class diagram of the this system.

\begin{figure}[t]
	\centering
	\includegraphics[scale=0.5]{diagrams/PerceptionClassDiagram.png}
	\caption{UML class diagram of the perception system.}
	\label{fig:percumlcd}
\end{figure}

For illustration, consider these examples. The perception PitBelow extends BoolPerception, with a unique index of 8 and implements \emph{apply} to return 1 if there is a pit directly below Mario and 0 otherwise. MarioMode extends BytePerception, with a unique index of 0 and a limit of 2, with \emph{apply} returning 0 if Mario is \textbf{small}, 1 for \textbf{big} and 2 for \textbf{fire}.

This approach keeps all the information about perceptions contained in one file. The number of perceptions is easily extended by adding more case objects. Furthermore, it allows the Observation and Conditions classes to be implemented as fixed length byte vectors, with the vector's index matching the perception's unique index field. With use of Scala's implicit and extractor functionality, building the Observation and validating the Conditions vectors is type safe and concise:


\begin{minipage}{0.9\linewidth}
\centering
\begin{lstlisting}[language=scala]
val observationVector = 
    Vector.tabulate(Perception.NUMBER_OF_PERCEPTIONS) {
        n: Int => n match {
            // This retrieves the perception object with the index
            // that matches n. 
            case Perception(perception) => perception(environment)
        }
    }
\end{lstlisting}
\end{minipage}

\begin{minipage}{0.9\linewidth}
\centering
\begin{lstlisting}[language=scala]
def validateConditions(conditionsVector: Vector[Byte]): Boolean = {
    conditionsVector.zipWithIndex.forall {
        case (b: Byte, Perception(perception)) => perception match {
            case boolP : BoolPerception => 
                (b == boolP.TRUE) || (b == boolP.FALSE) 
                    || (b == DONT_CARE)
            case byteP : BytePerception => 
                ((0 <= b) && (b <= byteP.limit)) || (b == DONT_CARE)
        }
        case _ => false 
    }
}

\end{lstlisting}
\end{minipage}

Notice that no information about specific concrete perceptions is required, enforcing the open-closed design principle and allowing perceptions to be added without need to alter this code.

\subsubsection{Perceiving the Environment}

The \emph{Environment} interface contains several individual methods that report Mario's situation. Therefore implementing Perceptions that concerned Mario (e.g. MarioMode, MovingX etc.) was trivial.

Perceptions pertaining to enemies, obstacles or pits was more challenging. Environment provides two 19x19 arrays, one for enemies and one for terrain. Centred around Mario, each array element represents a `square' (16 pixels) of the level scene. The value of an element marks the presence of an enemy or terrain square. 

Enemy and obstacle perceptions pass the relevant array, a lambda test function, coordinates for a box segment of the array and a boolean to a helper function. This function uses Scala's for loop comprehensions to search through the box, applying the lambda to each element, returning the boolean parameter if the lambda returns true at least once. In this way it is easy to search for an enemy or obstacle in a box relative to Mario. Pits work in a similar way, but declare columns instead. If there is no terrain in the column below Mario's height, then it is considered a pit. Take EnemyLeft for example:

\begin{minipage}{0.9\linewidth}
\centering
\begin{lstlisting}[language=scala]
case object EnemyLeft extends BoolPerception(3) {
    // Minus = (Up,Left) | Plus = (Down,Right)
    val AREA_UL = (-2,-2); val AREA_BR = (1, -1);
  
    def apply(environment: Environment): Byte = {
        if(Perception.enemyInBoxRelativeToMario(environment, AREA_UL, AREA_BR)) 
            1 else 0  
    }
}
\end{lstlisting}
\end{minipage}

\begin{minipage}{0.9\linewidth}
\centering
\begin{lstlisting}[language=scala]
def enemyInBoxRelativeToMario(
          environment: Environment, 
          a: (Int, Int), b: (Int, Int)): Boolean = {
    val enemies = environment.getEnemiesObservationZ(2);
    val test = (grid: Array[Array[Byte]], tup: Tuple2[Int, Int]) =>{
         val x = grid(tup._1)(tup._2)
         x == 1
    }
    checkBox(enemies, test, getMarioPos(environment), a, b, true)
}
\end{lstlisting}
\end{minipage}

\begin{minipage}{0.9\linewidth}
\centering
\begin{lstlisting}[language=scala]
def checkBox(grid: Array[Array[Byte]], 
               test: (Array[Array[Byte]], (Int, Int))=>Boolean, 
               mario: (Int, Int), 
               a: (Int, Int), b: (Int, Int), 
               ret: Boolean): Boolean = {
    import Math.min
    import Math.max
    val relARow = min(grid.length-1, max(0, (a._1 + mario._1)))
    val relACol = min(grid(0).length-1, max(0, (a._2 + mario._2)))
    val relBRow = min(grid.length-1, max(0, (b._1 + mario._1)))
    val relBCol = min(grid(0).length-1, max(0, (b._2 + mario._2)))
    
    for {
      i <- min(relARow, relBRow) to max(relARow, relBRow)
      j <- min(relACol, relBCol) to max(relARow, relBCol)
      if (test(grid, (i, j)))
    }{
        return ret
    }
    !ret
}
\end{lstlisting}
\end{minipage}


\subsubsection{Observation, Conditions and Actions}

For clarity an Action class was created for use in the ruleset, with an adapter method to convert it into the boolean array expected in the Agent interface. As the Observation and Conditions classes were to be implemented as fixed length byte vectors, so was the Action class.

Action vectors have a fixed length of 4, where elements represent \textbf{left}, \textbf{right}, \textbf{jump} and \textbf{speed} respectively. Observation vectors have a fixed length equal to the number of perceptions and hold the byte returned by each Perception's apply function. Conditions have the same length and hold data in the same range as Observation, the condition on each perception therefore being that the corresponding element in the observation be equal. They have one additional possible value, {\footnotesize DONT\_CARE} (equal to byte -1), which represents that no condition be placed on that perception.

Instead of implementing these classes as wrappers for \emph{Vector[Byte]}, which can be inefficient and overly verbose, type aliases were used. This allowed each class to be referred to explicitly (rather than just by variable name), which provides readability and type safety, whilst still having direct access to the list processing methods included in the Vector class. They were declared on the agent framework's package object, making them accessible package wide. An object with static data and factory methods was included for each.

For example, this allowed Observation to be used as such:

\begin{minipage}{0.9\linewidth}
\centering
\begin{lstlisting}[language=scala]
abstract class ObservationAgent extends Agent {
    ...
    // Using factory method for a blank observation
    var observation: Observation = Observation.BLANK
    ...
    def integrateObservation{env: Environment}: Unit = {
        // Using the Observation factory method 
        // to build a new observation
        observation = Observation(env)
        
        if (printObservation) {
            // Using Vector method foreach directly
            observation.foreach { 
                b:Byte => print(" ~ " + b)
            }
        } 
        ...
    }
    ...
 }

\end{lstlisting}
\end{minipage}


\subsubsection{Rules and Rulesets}

Having both Conditions and Action implemented as byte vectors allows rules to be represented in the same way. Each rule is simple the concatenation of the Conditions and Action vectors. The rule vector are fixed length as both Conditions and Action are. Moreover, as rulesets contain just a list of rules, rulesets can be unambiguously represented by a single dimension byte vector. This allows rulesets not only to be persisted easily (as say a csv) but also gives the data representation needed for the evolutionary process.

In this case both Rule and Ruleset were implemented as wrapper classes for \emph{Vector[Byte]} and \emph{Seq[Rule]} respectively. Ruleset also holds a default Action, which is used if no rule matches the environment.

The semantic reasoner of the rule system is split across both classes. In Rule, the \emph{scoreAgainst} method is passed the observation and produces a score by looping through the conditions and adding 1 if the condition and observation match in value and 0 if the condition is {\footnotesize DONT\_CARE}. If a mismatched condition is found, the method immediately returns with -1. It is implemented tail recursively to provide maximum efficiency.

\begin{minipage}{0.9\linewidth}
\centering
\begin{lstlisting}[language=scala]
def scoreAgainst(observation: Observation): Int = {
    val conditions = ruleVector.slice(0, Conditions.LENGTH)
    @tailrec
    def scoreRecu(i: Int, sum: Int = 0): Int =  {
        if (i == Conditions.LENGTH) sum
        else conditions(i) match {
            case Conditions.DONT_CARE => scoreRecu(i+1, sum)
            case b if b == observation(i) => scoreRecu(i+1, sum+1)
            case _ => -1
        }
    }
    scoreRecu(0)
}
\end{lstlisting}
\end{minipage}

In Ruleset, the \emph{getBestAction} is passed the observation and returns an action boolean array. Using tail recursion it performs a fold operation on its rules, saving and returning the best scoring rule (preferring earlier rules when tied). If no rule gets a score of 0 or above then the default action is returned.

\begin{minipage}{0.9\linewidth}
\centering
\begin{lstlisting}[language=scala]
def getBestExAction(observation: Observation): Array[Boolean] = {
    
    @tailrec
    def getBestRuleRecu(rs: Seq[Rule], best: Option[Rule] = None, bestScore: Int = 0): Option[Rule] = 
        rs match {
          case Nil => best
          case (r +: ts) => {
              val newScore = r.scoreAgainst(observation)
              if (newRuleBetter(bestScore, newScore))
                  getBestRuleRecu(ts, Some(r), newScore)
              else
                  getBestRuleRecu(ts, best, bestScore)
        }
    }
    
    getBestRuleRecu(rules) match {
        case None => defaultAction.toBooleanArray
        case Some(r) => r.getAction.toBooleanArray
    }
}
\end{lstlisting}
\end{minipage}


\subsubsection{Persistence}

Agent's are persisted by persisting their ruleset. Rulesets are persisted in single line csv files. An IO helper object is passed an agent, extracts its ruleset and requests its vector representation, writing each byte separated by a comma. On reading an agent file, it constructs a byte vector. This byte vector is passed to the Ruleset's factory method, which groups the vector by rule length to form the rule sequence.

%-----------------------------------------------------------------------
% Testing
%-----------------------------------------------------------------------

\subsection{Testing}

Due to the agent modules heavy reliance of the Environment interface the use of a mocking facility was required. The ScalaMock testing library was adding to the project to provide this.

Perceptions were unit tested individually, using white-box approach (due to the inclusion of mocking). Each test stubbed the Environment interface, instructing it to return a specific value (or array) for the relevant call, and testing that the perception echoed or processed it correctly. This allowed Perceptions to be tested independently of the game engine and provided test coverage. However, as there was very little documentation Environment interface, expected return values had to be investigated manually and edge cases could have easily been missed.

Rulesets (and Rules) were tested with a largely black-box end-to-end style. This was required due to the reliance of type aliases. Fixed rulesets were constructed to verify that the \emph{getAction} method returned the expected action based on a fixed observation. Mocking of individual rules was not used in case the Rule class was altered to be a type alias instead of a wrapper class.

These tests were added to Maven's build lifecycle, and hence run on every project build.

%-----------------------------------------------------------------------
% Handcrafted
%-----------------------------------------------------------------------

\subsection{Handcrafted Agents}
\label{subsec:hca}

For the purpose of evaluation and comparison three handcrafted rulesets were created for the agent framework.

\begin{description}
	\item[Forward Jumping] This ruleset commands the agent to jump whenever it can, regardless of its surroundings. It contains a single rule and a default action. It is a blind agent that does not take advantage of the framework, however it is surprising effective. An analogous agent was used for comparisons in the 2009 Mario AI Competition and was found to score higher than many entrants. The learning process will aim to discourage this behaviour as it is neither interesting or optimal.
	\item[Simple Reactive] This agent only jumps when it detects an enemy, obstacle or pit in its way. It contains 5 rules and defaults to moving right at speed. This agent makes better use of the framework, but still does not use all of its perceptions. Its behaviour is more interesting, however it tends to score similarly to the forward jumping agent. Despite low attainment, a learnt agent that behaves in a similar way will be evaluating more favourably as it is using more of the agent's perceptions.
	\item[Complex] This agent is the most interesting and highest scoring of the three. It has several different behaviours and builds off of the simple reactive agent. It contains 18 rules and makes use of all perceptions except MarioMode and EnemyLeft. Its behaviour was investigated at length and as such will form a good comparison to the final learnt agent. Effective evolved behaviours unconsidered in this agent's creation are a sign of the validity of the learning process. 
\end{description}
Creation of the handcrafted rulesets also informed additions and alterations to the agent's perceptions. For example, it was originally difficult to create an effective pit strategy and so PitAhead was changed from a BoolPerception to a BytePerception, returning 3 values representing \textbf{none}, \textbf{far} and \textbf{close}.
!APPENDIX!


%___________________________________________
%*************************************************************
% Level Playing
%___________________________________________
%^^^^^^^^^^^^^^^^^^^^^^^^^^^^^^^^^^^^^^^^^^^^^^^^^^^

\section{Level Playing Module}

The main purpose of the level playing module is to evaluate the fitness of agents during the learning process. An effective learning process needs a diverse testbed, thus the level playing module must be deterministic, highly configurable and able to provide variety. 

%-----------------------------------------------------------------------
% Design
%-----------------------------------------------------------------------

\subsection{Design}

\begin{figure}[t]
	\centering
	\includegraphics[scale=0.5]{diagrams/PlayingClassDiagram.png}
	\caption{UML class diagram of the level playing module.}
	\label{fig:pumlcd}
\end{figure}

In the benchmark the heart of the game engine is the \emph{MarioEnvironment} class. It is responsible for calling the \emph{LevelGeneration} class, updating the scene each tick with an action, and reporting the scene through the Environment interface. Parameters controlling level generation and playing are contained in the \emph{MarioAIOptions} class. The \emph{BasicTask} class controls the game loop in conjunction with an agent. It initialises the MarioEnvironment class with options, then runs through the loop, commanding a tick, passing the environment to the agent and finally requesting an action and passing it to the MarioEnvironment instance. Statistics pertaining to the agents performance are stored in MarioEnvironment in the \emph{EvaluationInfo} class, which are cloned to BasicTask at the end of the level. Fitness can be requested from EvaluationInfo with the optional parameter, a \emph{SystemOfValues} instance, which contains the multipliers for various measurements.

The MarioAIOptions class contains several useful parameters. They include: a integer seed, which is passed to the level generator's RNGs; a boolean that toggles visualisation; integers that determine level difficulty, length and time-limit; and booleans that toggle enemies, pits and blocks. The SystemOfValues class contains multipliers for  distance achieved, whether or not the level was completed, the amount of time left on completion as well as many others.

\vspace{\baselineskip}

The level playing module will extend this system. Its primary objective will be to allow for multiple levels (episodes) to be played with options that update as prescribed by some parameter class. Upon completion it will produce a fitness based on all levels played. Furthermore, it will allow for the injection of different agents and level seeds, to allow different agents to play the same options without rebuilding the instance. A UML class diagram for the module can be found in Figure \ref{fig:pumlcd}.

Level options are stored in the \emph{MWLevelOptions} class, which acts as an adapter for the MarioAIOptions class. Each new level these will be updated by the dedicated \emph{MWLevelOptionsUpdater} class. The \emph{EvaluationMultipliers} class is an adapter for the SystemOfValues class, which is used to calculate the fitness at the end of a sequence of levels. These classes are separated from the the game playing task and are included using constructor dependency injection. This helps to decouple the system and improve the ability to modify and test.

Both MWLevelOptions and EvaluationMultipliers are designed to be data classes, which provides determinism and thread safety, as well as easy persistence. MWLevelOptionsUpdater's qualities in this regard are Implementation details and are discussed in a following section. The full list of parameters held in these classes can be found in the Appendix !REF!.


%-----------------------------------------------------------------------
% Implementation
%-----------------------------------------------------------------------

\subsection{Implementation}

\subsubsection{Parameter Classes}
\label{subsec:paramclasses}

The MWLevelOptions and MWEvaluationMultipliers classes were implemented as data classes in an immutable builder pattern style. Each field has a \emph{withField(field)} method that returns a cloned instance with that field changed. This affords a concise, declarative style, whilst maintaining immutability. MWEvaluationMultipliers has a implicit converter to a SystemOfValues instance, which is required for evaluation in the benchmark. However, a similar approach was not possible for converting MWLevelOptions to a MarioAIOptions instance (required for the game-engine). MarioAIOptions hold more parameters than MWLevelOptions (e.g. agent and level seed), therefore the adapter function takes both the current MarioAIOptions and a MWLevelOptions instance as parameters and updates the former by overriding the corresponding fields with values from the latter. A full list of field for both the MWLevelOptions and MWEvaluationMultipliers classes can be found in the appendix (\ref{}). !APPENDIX!

MWLevelOptions was not implemented as a class. Instead, MWBasicTask expects a lambda function. This lambda takes the episode number and the current MWLevelOptions, returning an updated set of options. MWLevelOptions builder structure ensures this is always a new instance, and hence maintains immutability. The inclusion of the episode number allows the function to remain deterministic. Moreover, with the ability to build closures in Scala, this lambda can be built from data (for example, a list of options, where indexes relate to episode number).

\subsubsection{Persistence}
\label{subsec:evalparams}

The primary use of the level playing module is during the evaluation stage of the learning process. Hence, it was decided to use ECJ's parameter file system to persist level playing parameters, which allows them to be written in the same file as the rest of the learning parameters.

ECJ's parameter system builds upon Java's Properties file system. From a parameter file (formatted in the Java Properties format) a ParameterDatabase can be built, from which specific parameters can be requested using the Parameter class. This system was used to persist the two parameter classes, MWLevelOptions and MWEvaluationMultipliers; the level options update lambda data; and other level playing data such as number of levels (episodes) and base level seed. For example, the following lines would set the number of levels to 10 and the base difficulty to 5:

\begin{minipage}{0.9\linewidth}
\centering
\begin{lstlisting}
          level.num-levels = 10
level.base.difficulty-num = 5
\end{lstlisting}
\end{minipage}

A static utility class EvaluationParamsUtil was created to handle the reading of the level playing data from these files. A ParameterDatabase is built and passed to utility class, which builds the required parameter class. For MWLevelOptions and MWEvaluationMultipliers it searches for the prefixes `{\ttfamily level.base}' and `{\ttfamily mult}' respectively and then looks for suffixes corresponding to specific fields. If a field's suffix is not found then it is initialised to a default value (which is always zero for MWEvaluationMutlipliers).

The update lambda is built as a closure on a collection of Map instances, one for each field in MWLevelOptions. For each {\ttfamily n} from 0 to the number of levels (episodes), the utility function looks for the prefix `{\ttfamily level.\{n\}}', which is used to hold the update for episode \textbf{n}. For each field a Map is built and using the same suffixes as for MWLevelOptions, key-value pairs are added mapping \textbf{n} to the value found. When the update lambda is called, it consults these maps and updates the MWLevelOptions with the new value if one is found for the current episode number.

For example, if the parameter file contained the following lines:

\begin{minipage}{0.9\linewidth}
\centering
\begin{lstlisting}
   level.num-levels = 4
 level.base.enemies = false
    level.1.enemies = true
    level.3.enemies = false
\end{lstlisting}
\end{minipage}

Then enemies would be off for the first episode, on for the second and third and off again for the fourth and final episode.

\subsubsection{The Episode Loop}

The entry point of the level playing module is the MWEvaluationTask class, which extends the abstract MWBasicTask class. MWEvaluationTask is instantiates with a base set of options (as MWLevelOptions), a MWEvaluationMutlipliers instance and an update lambda, as well as the number of episodes (levels) to run. The agent and base level seed can be injected with the \emph{withAgent(agent)} and \emph{withLevelSeed(seed)} methods (which also reset of evaluation information).

\begin{minipage}{0.9\linewidth}
\centering
\begin{lstlisting}[language=scala]
class MWEvaluationTask(val numberOfLevels: Int, 
                       val evalValues: MWEvaluationMultipliers, 
                       override val baseLevelOptions: MWLevelOptions, 
                       override val updateOptionsFunc: (Int, MWLevelOptions) => MWLevelOptions)
                           extends MWBasicTask("MWMainPlayTask", baseLevelOptions, updateOptionsFunc, visualisation, args) with EvaluationTask {

    private var baseLevelSeed: Int = 0;
    
    override def nextLevelSeed(episode: Int, lastSeed: Int) =  {
        (3*episode) + lastSeed
    }    
        
    override def evaluate: Int = {
        doEpisodes
        localEvaluationInfo.computeWeightedFitness(evalValues)
    }
  
    override def withAgent(agent: Agent): MWEvaluationTask = {
        super.injectAgent(agent, true)
        this
    }
\end{lstlisting}
\end{minipage}     
    
\begin{minipage}{0.9\linewidth}
\centering
\begin{lstlisting}[language=scala]   
    override def withLevelSeed(seed: Int): MWEvaluationTask = {
        baseLevelSeed = seed
        super.injectLevelSeed(seed, true)
        this
    }
}
\end{lstlisting}
\end{minipage}  

On calling the \emph{evaluate()} method, the number of levels is passed to the \emph{doEpisodes(numberOfEpisode)} (a superclass method), which loops as follows:

\begin{minipage}{0.9\linewidth}
\centering
\begin{lstlisting}[language=scala]

def doEpisodes(amount: Int): Unit = {
    @tailrec
    def runSingle(iteration: Int, prevOptions: MWLevelOptions, disqualifications: Int): Int = {
        if (iteration == amount) { 
            disqualifications
        } else {
            // Calls the update lambda to get episodes set of options
            val newOptions = updateOptions(iteration, prevOptions)
            
            // Converts options to class required for game-engine
            val marioAIOptions = MWLevelOptions.updateMarioAIOptions(super.options, newOptions)
            
            // Updates the level seed (which is set to increase each episode by MWEvaluationClass)
            // Agent instance is already being held here
            marioAIOptions.setLevelRandSeed(nextLevelSeed(iteration, marioAIOptions.getLevelRandSeed))
            
            // Resets the evaluation information in the super class
            super.setOptionsAndReset(marioAIOptions)
\end{lstlisting}
\end{minipage}

\begin{minipage}{0.9\linewidth}
\centering
\begin{lstlisting}[language=scala]
            // Generates and runs the level using the super class
            // Returns true if the agent was not disqualified (took too long to return an action)
            val notDisqualified: Boolean = runSingleEpisode(1)
            val disqualification: Int = if (!notDisqualified) 1 else 0
       
            // Update the evaluation information for the entire run
            // (as the super classes evaluationInfo gets reset every level
            updateLocalEvaluationInfo(super.getEvaluationInfo)
            
            // Loop
            runSingle(iteration+1, newOptions, disqualifications + disqualified)
        }
    }
    
    // Sets the base options
    super.setOptionsAndReset(MWLevelOptions.updateMarioAIOptions(options, baseLevelOptions))
    disqualifications = runSingle(0, baseLevelOptions, 0)
}
\end{lstlisting}
\end{minipage}


Before every episode the options are updated by the lambda, updating the base options in the first episode. A new level seed is also requested from the MWEvaluationClass, which simply increases it each episode. These updates are converted and added to a MarioAIOptions instance and passed to the superclass in the benchmark. A single level is then generated and played using the superclass. Evaluation information is added to MWBasicTask's local evaluation information (as the former is reset every episode) and the function loops tail recursively.

When the number of loops equals the number of levels the function exits. At this point the \emph{evaluate()} method requests a fitness from the local evaluation information, passing in the evaluation multipliers, which is then returned.

%-----------------------------------------------------------------------
% Benchmark Edit
%-----------------------------------------------------------------------

\subsection{Modifying the Benchmark}
\label{subsec:enginemod}

Preliminary runs of the benchmark software revealed several issues and defects. As the learning algorithm would run over several generations, any errors could halt it prematurely, which could be costly in terms of project time keeping. In order to address this a Java project was created from the benchmark's source code and fixes were made. This code was packaged with Maven and included as a dependency in the main project.

Several minor exceptions were caught or addressed, however the two largest issues failed `quietly'. They concerned the LevelGeneration class and surrounded enemy and pit generation in regards to level difficulty. Fixes were made to ensure level difficulty scaled more consistently.

\subsubsection{Enemy Generation}

In observing the benchmark generated levels it was apparent that enemy density was very high, even on lowest difficulties. Examining the LevelGeneration class revealed this to the result of what was probably unintended behaviour. 

Levels are generated in zones of varying length. Quite often, a zero length zone is created, which has no effect on terrain. However, enemies were still being added to these zones, creating very high density columns of enemies during levels. This was addressed, with the addition of a more gentle enemy count curve and better spacing.

\subsubsection{Pit Generation}

Another apparent shortcoming was pit length. Pits were only of two sizes, small and very large, and after a certain level difficulty were always very large. Although this was intended behaviour, comments from the original developers suggest it was a placeholder for a more sophisticated system. An edit was made to scale maximum pit length on level difficulty. Each pit's length is chosen probabilistically on a bell curve, which is shifted by level difficulty.


%-----------------------------------------------------------------------
% Testing
%-----------------------------------------------------------------------

\subsection{Testing}

The benchmark's BasicTask class is tightly coupled to the MarioEnvironment, which means that we were unable to mock the game-engine for testing. This problem extends to the MWEvaluationTask class.

However, the decoupling of the parameter classes from MWEvaluationTask means that the persistence section of the module can easily be tested. The EvaluationParamUtil class is white-box unit tested by stubbing the ParameterDatabase interface and verifying the contents of the parameter classes requested. For example:

\begin{minipage}{0.9\linewidth}
\centering
\begin{lstlisting}[language=scala]
"getEvaluationMutlipliers" should "return eval mults from databade and zero otherwise" in {
    val base = blankParam.push(EvaluationParamsUtil.P_EVAL_BASE)
    (pdStub.exists _) when(base, *) returns(true)
    
    (pdStub.exists _) when(base.push(EvaluationParamsUtil.P_COINS), *) returns(true)
    (pdStub.getIntWithDefault _) when(base.push(EvaluationParamsUtil.P_COINS), *, *) returns(10)
    
    (pdStub.exists _) when(base.push(EvaluationParamsUtil.P_KILLED_BY_SHELL), *) returns(true)
    (pdStub.getIntWithDefault _) when(base.push(EvaluationParamsUtil.P_KILLED_BY_SHELL), *, *) returns(200)
    
    (pdStub.exists _) when(base.push(EvaluationParamsUtil.P_DISTANCE), *) returns(true)
    (pdStub.getIntWithDefault _) when(base.push(EvaluationParamsUtil.P_DISTANCE), *, *) returns(1)
    
    val evalMults = EvaluationParamsUtil.getEvaluationMutlipliers(pdStub, base);
    
    assert(evalMults.coins == 10)
    assert(evalMults.killedByShell == 200)
    assert(evalMults.distance == 1)
    assert(evalMults.win == 0)
    assert(evalMults.kills == 0)
    ...
}

\end{lstlisting}
\end{minipage}


%-----------------------------------------------------------------------
% Comparator Task
%-----------------------------------------------------------------------

\subsection{Comparator Task}

In order to quantifiably compare agents a competitive set of evaluation task options was created, modelled on those used during the final evaluation stage of the 2010 Mario AI Competition.

Agents play 512 levels, spread equally over 16 difficulty levels (0 to 15). Options such as enemies, pits, blocks etc. are periodically turned off for a level. Length is varies greatly from level to level, with the time-limit being adjusted accordingly. Evaluation multipliers reward all possible positive statistics, such as enemy killed, coins collected and distance travelled. A full view of the comparator task can be found in the appendix (\ref{}). !APPENDIX!

The scores and other statistics attained by the three handcrafted agents playing the evaluation task, on seed 10, can be found in Table \ref{tab:hceval}.

\begin{table}
  \begin{adjustwidth}{-0cm}{-0cm}
  \begin{center} \small
    \begin{tabular}{ | l | c | c | c | c | c |}
    \hline
    & \textbf{Total} & \textbf{Levels} & \textbf{Enemies} & \Tstrut \\
    \textbf{Agent} & \textbf{Score} & \textbf{Completed} & \textbf{Killed} & \textbf{Distance} \Bstrut \\ \thickhline
    \textbf{Complex} & 1,817,195 & 171 (33\%) & 1498 (8\%) & 63,894 (48\%) \\ \hline
    \textbf{Simple Reactive} & 1,095,287 & 88 (17\%) & 590 (3\%) & 43,286 (32\%) \\ \hline
    \textbf{Forward Jumping} & 954,640 & 76 (15\%) & 677 (3\%) & 36,980 (28\%) \\ \hline

    \end{tabular}
  \end{center}
  \end{adjustwidth}
  \caption{\small Competitive statistics from handcrafted agents playing the evaluation task with a seed of 10.}
  \label{tab:hceval}
\end{table}

%___________________________________________
%*************************************************************
% Learning
%___________________________________________
%^^^^^^^^^^^^^^^^^^^^^^^^^^^^^^^^^^^^^^^^^^^^^^^^^^^

\section{Learning}

The learning module's goal is to utilise the ECJ library to evolve a population of byte vectors (representing agent rulesets). This is achieved through extending ECJ's classes, in conjunction with a parameter file. The ECJ library is an extensive collection of learning processes and, due to the limitations of this report, only the relevant sections, and the extensions thereof, are described. An important part of the project was adjusting the parameter file in reaction to previous learning runs, these revisions are discussed in section \ref{subsec:learnparam}. However, some parameters were deemed fundamental and stayed invariant throughout this process; these are described, alongside the general evolutionary approach, in the following section.

%-----------------------------------------------------------------------
% Design
%-----------------------------------------------------------------------

\subsection{Evolutionary Approach}
\label{subsec:learndes}

Similar to the approach taken in developing the REALM agent (described in section \ref{ssec:realm}), this project uses a  $(\mu  + \lambda)$ evolution strategy. However, unlike REALM, it focuses purely on mutation, with no crossover between individuals (a technique not traditionally applied with ESes \cite[p.~]{ecj-tut3}).

Evolution runs for 1000 generations with a population size of 50. The truncation selection size for our ES ($\mu$) is set to 5. Hence, the top 5 fittest individuals are chosen each generation. These are cloned and mutated to form 45 ($\lambda$) new individuals, which are then joined by the original 5 for the next generation.

Each individual's genome is a byte vector of length 300, representing a ruleset containing 20 rules. The default action is fixed over all rulesets and is specified by the parameter file.

Each element of the genome (a gene) has several mutation parameters associated to it. During the mutation phase each gene will be mutated with the probability prescribed by its mutation probability. Each gene can be mutating in one of two ways. The first will simply choose a random byte between its minimum and maximum values (inclusive), known as \textbf{reset} mutation. The second will favour one particular byte value, with a certain probability, which is known as \textbf{favour} mutation. Which method of mutation a gene undergoes, as well as the favoured byte and its probability are all contained in the individual gene's parameters, allowing for very fine grained control over the mutation process.

The evaluation stage takes each individual's genome and constructs an agent. The individual's fitness is calculated by passing the agent to the MWEvaluationTask, which has been initialised with a fixed set of level playing parameters decreed by the parameter file. These parameters have a custom set of evaluation multipliers and describe a task of 10 levels, varying level options for each level. Each generation builds this task with its own unique level seed to avoid over-fitting to one set of generated levels.

Statistics for average and best fitness are logged each generation. Level seed is also reported, to allow direct comparison between the learning process and the hand-crafted agents. After the final generation a selection of agents are written to agent files, including the best individual of the final generation and the best individual over all generations.

%-----------------------------------------------------------------------
% Extending the ECJ Library
%-----------------------------------------------------------------------

\subsection{The ECJ Library}

\begin{figure}[t]
	\centering
	\includegraphics[scale=0.5]{diagrams/ECJTopLevel.png}
	\caption{A top-level UML class diagram of the operation facilities in EvolutionState, taken from \cite[p.~10]{ecj-manual}.}
	\label{fig:ecjop}
\end{figure}

\begin{figure}[t]
	\centering
	\includegraphics[scale=0.5]{diagrams/ECJTopLevelData.png}
	\caption{A top-level UML class diagram of ECJ's data objects, taken from \cite[p.~11]{ecj-manual}.}
	\label{fig:ecjdata}
\end{figure}

\begin{figure}[t]
	\centering
	\includegraphics[scale=0.5]{diagrams/LearningClassDiagram.png}
	\caption{A simplified UML class diagram of the Learning module's extension to ECJ.}
	\label{fig:lucl}
\end{figure}


The primary method for specialising ECJ for a particular use case is to extend the classes it provides, and specify these subclasses, alongside other pertinent values, in the parameter file. Nearly every method in ECJ's classes is passed the pseudo-global EvolutionState singleton, which holds instances of all other classes in use. A top-level view of ECJ's structure can be found in figures \ref{fig:ecjop} and \ref{fig:ecjdata}.

The learning module will be built on top of the ECJ library, with a view to adhering to its style and ideology. For our approach, several of ECJ's classes did not need to extended, and simply specifying their default implementations in the parameter file was enough. The classes of interest to this project are the following: \emph{Species}, \emph{Individual}, \emph{SelectionMethod}, \emph{Breeder}, \emph{BreedingPipeline}, \emph{Evaluator}, \emph{Problem} and \emph{Statisitics}. A simplified class diagram of the learning module is included as figure \ref{fig:lucl}.

ECJ provides the \emph{ByteVectorIndividual} class, which contains a Java array of \emph{byte}s, this is used to hold individual byte vector genomes. It also decrees the use of the \emph{IntegerVectorSpecies} class, which holds the gene parameters (e.g. minimum value, mutation probability). Our design first extends this class to the \emph{DynamicParameterIntegerVectorSpecies} class, which holds a subclass instance of another new class, the \emph{DynamicSpeciesParameters} class. The specific subclass to be used is specified in the parameter file. The purpose of this class is to provide a facility to override the min, max and mutation probability properties of each gene directly from the agent framework (rather than the parameter file). This class is then further extended to the \emph{RulesetSpecies} class, which holds gene parameters pertaining to favour mutation. This class also extends the parameter file's vocabulary to allow gene parameters to specified on a \textbf{condition} or \textbf{action} (the composite parts of a rule) basis. For example, the favour byte for all genes that represent conditions can be set as follows in the parameter file:

\begin{minipage}{0.9\linewidth}
\centering
\begin{lstlisting}
   ...species.condition.favour_byte 	= -1
\end{lstlisting}
\end{minipage}


The \emph{SelectionMethod} and \emph{Breeder} classes are responsible for selecting individuals (based on fitness) to clone and pass to the breeding pipeline. ECJ natively supports evolution strategies and so it is enough to specify these classes to be ECJ's \emph{ESSelection} and \emph{MuPlusLambda} classes. The \emph{BreedingPipeline} extension, \emph{RulesetMutationPipeline}, performs the mutation of individuals. For each gene in a individual's genome it requests the corresponding mutation parameters from the species class. It then clones and, if required, mutates the gene with either favour or reset mutation. 

The \emph{Evaluator} class is responsible for threading and batching the individuals to have their fitness calculated by cloned instances of the \emph{Problem} class. Hence, only the Problem class is extended, to the \emph{AgentRulesetEvaluator} class. On initialisation this class reads the the level options from the parameter file using the EvaluationParamsUtil class described in section \ref{subsec:evalparams}. Every generation, agents are constructed from the individuals of the population and passed, with the options and the level seed, to the MWEvaluationTask class. This returns a score, which is attached to the individual as its fitness.

Generational statistical reporting is handled by an extension of the \emph{Statistics} class, which will output to log files specified in the parameter file.



%-----------------------------------------------------------------------
% Implementation
%-----------------------------------------------------------------------

\subsection{Implementation}

Extensions to the ECJ library were implemented in Scala, organised into a package and situated in the same project as the agent framework and level playing modules. This was to allow the level playing parameters to be read through ECJ's ParameterDatabase class (described in section \ref{subsec:evalparams}).

\vspace{\baselineskip}

Nearly every class in the ECJ library implements a \emph{setup(...)} method, which is used initialise an instance. This method is passed the EvolutionState singleton, from which it can access the ParameterDatabase, which holds the contents of the parameter file. Classes can poll the ParameterDatabase by constructing a \emph{Parameter} instance (in a composite pattern) with string keys.

\subsubsection{Modifying the ECJ Library}
\label{subsec:ecjmod}

During the implementation process it became clear the the proposed species design was problematic due to the setup process of IntegerVectorSpecies and its superclass \emph{VectorSpecies}. The setup method reads in values for gene minimum, maximum and mutation probability and then calls for the \emph{Individual} prototype to be built. This means that any subclass of IntegerVectorSpecies cannot decide the values of the gene minimum, maximum and muation probability. If they are written before calling setup on the superclass (which is required) they will be overridden, if they are written after the calling setup on the superclass they will not be built into the prototype. To address this the VectorSpecies class was modified to include the \emph{prePrototypeSetup} method, which is called immediately before constructing the prototype in the setup method. Subclasses can extend this method to overwrite any properties set by IntegerVectorSpecies and VectorSpecies.

\begin{minipage}{0.9\linewidth}
\centering
\begin{lstlisting}[language=java]
public void setup(final EvolutionState state, final Parameter base) {
    // Add min, max, mutation probability etc.
    ...

    // Allow subclasses to override values
    prePrototypeSetup(state, base, def);
    state.output.exitIfErrors();          

    // NOW call super.setup(...), which will in turn sets up the prototypical individual
    super.setup(state,base);
}

// new method added
protected void prePrototypeSetup(final EvolutionState state, final Parameter base, final Parameter def) {
    //None by default, for subclasses
}
\end{lstlisting}
\end{minipage}

\subsubsection{Species}

IntegerVectorSpecies and VectorSpecies store gene parameters (e.g. min, max, mutation probability) as arrays of equal length to the genome, matching by index. These arrays are built during the setup method by polling the ParameterDatabase for the suffixes `{\ttfamily min-value}', `{\ttfamily max-value}' and `{\ttfamily mutation\-prob}'. The parameter file can specify them globally, in segments or by individual gene (by index). For segments and single indexes the \emph{loadParameterForGene} method is used, which takes the gene index and a prefix parameter (e.g. segment number or index). It builds a Parameter instance by appending the gene param suffixes to the passed prefix. It requests the Parameter's value from the ParameterDatabase. If a value is found, it sets it as the element of the corresponding gene parameter array at the given index..

\subsubsection*{\hspace{6pt}Dynamic Parameters}

The DynamicParametersIntegerVectorSpecies performs the same task as described above, but obtains the values from a specified class rather than the parameter file. This class must be a subclass of the DynamicSpeciesParameters class. DynamicSpeciesParameter define methods for global, segment and index gene parameters, each return an Option on the value. The default implementation returns None, allowing subclasses to override only the methods they want to be used in the learning run. For out learning run we used the RulesetParams class, which implemented the \emph{minGene(index)} and \emph{maxGene(index)} methods. These methods return the min and max value for a particular gene, with the values being obtained via the agent framework:

\begin{minipage}{0.9\linewidth}
\centering
\begin{lstlisting}[language=scala]
def getIndexType(index: Int): IndexType = (index % ruleLength) match {
    case n if n < conditionLength => Condition
    case _ => Action
}
override def maxGene(index: Int): Option[Int] = getIndexType(index) match {
    case Action => Some(Math.max(MWAction.ACTION_FALSE, MWAction.ACTION_TRUE))
    case Condition => (index % ruleLength) match { 
        case Perception(perception) => perception match {
            case bp : BoolPerception => Some(Math.max(bp.TRUE, bp.FALSE))
            case ip : BytePerception => Some(ip.limit.toInt)
        }
        ...
    }
}
\end{lstlisting}
\end{minipage}

The setup method of DynamicParametersIntegerVectorSpecies reads the subclass name (in this case RulesetParams) from the parameter file and instantiates it. The prePrototypeSetup method is then used to write the values (overriding those set by IntegerVectorSpecies). It calls each of DynamicSpeciesParameters' methods, if None is returned no action is taken, but if a value is return it is written to the parameter arrays.

\begin{minipage}{0.9\linewidth}
\centering
\begin{lstlisting}[language=scala]
override def prePrototypeSetup(state: EvolutionState, base: Parameter, default: Parameter): Unit = {
    if (dynamicParamsClassOpt.isDefined) {
        val dynamicParamsClass: DynamicSpeciesParameters
            = dynamicParamsClassOpt.get
      
        if (dynamicParamsClass.minGene.isDefined) {
            fill(minGene, dynamicParamsClass.minGene.get)
        }
        ...
        for(i <- 0 until genomeSize) {
    	        ....
            if (dynamicParamsClass.maxGene(i).isDefined) {
                maxGene(i) = dynamicParamsClass.maxGene(i).get
            }
            ...
        }
    }
    super.prePrototypeSetup(state, base, default)
}

\end{lstlisting}
\end{minipage}

\subsubsection*{\hspace{6pt}Ruleset Species}

The prePrototypeSetup method of RulesetSpecies checks that the DynamicSpeciesParameter class used is RulesetParams, which contains several utility functions. Adding to globally, by segment and by index methods for declaring gene parameters, this setup also looks for Parameter prefixes `{\ttfamily condition}' and `{\ttfamily action}'. If found, it uses a utility function to run loadParametersForGene on all indexes corresponding to the prefix. This method is also passed the prefix, in this way IntegerVectorSpecies specific parameters such as min and max can be set on a condition or action basis without needing to implement them directly. For example, take the following line in the parameter file:

\begin{minipage}{0.9\linewidth}
\centering
\begin{lstlisting}
...species.condition.mutation-prob = 0.1
\end{lstlisting}
\end{minipage}

The `{\ttfamily condition}' prefix is found by RulesetSpecies setup method, which indirectly loops through all indexes {\ttfamily x} that pertain to conditions:

\begin{minipage}{0.9\linewidth}
\centering
\begin{lstlisting}[language=scala]
override def prePrototypeSetup(state: EvolutionState, base: Parameter, default: Parameter): Unit = {
    super.prePrototypeSetup(state, base, default)
    ...
    if(state.parameters.exists(base.push(RulesetSpecies.P_CONDITION), default.push(RulesetSpecies.P_CONDITION))) {
        dpc.runOnIndexes(Condition, genomeSize){
            (x: Int, mod: Int) => {
                loadParametersForGene(state, x, base.push(RulesetSpecies.P_CONDITION), default.push(RulesetSpecies.P_CONDITION), "")
            }
        }
    }
}
\end{lstlisting}
\end{minipage}

It passes the condition prefix and the index to the loadParameterForGene method. In IntegerVectorSpecies, the suffix `{\ttfamily mutation-probability}' is added to the condition prefix and the value 0.1 is found, and set for all indexes {\ttfamily x} in the mutation probability array.

\vspace{\baselineskip}

RulesetSpecies also extends the loadParametersForGene method to read in the values for favour mutation from the parameter file. This is controlled by three arrays: favourMutation, a boolean array which holds whether or not a gene should be favour mutated; favourByte, a byte array holding the favoured byte; and favourProbability, the probability with which this byte will be chosen. By extending loadParametersForGene these parameters can be read in on condition or action bases, as well as by segment or individual index.


\subsubsection{Mutation}

To implement the mutation strategy, the RulesetMutationPipeline was created. It extends the BreedingPipeline and overrides the \emph{produce} method. It makes no use of the setup method as all mutation parameters are stored in the Species class. The breeding pipeline is threaded, and as such RulesetMutationPipeline is prototypes and cloned for each breed thread. The produce method is called to mutate batches of Individuals, which vary in size.

RulesetMutationPipelines produce method first calls produce on its source (in this case the ESSelection class), which fills the array of individuals to be mutated. It verifies that these individuals are ByteVectorIndividuals and their Species is RulesetSpecies. It then clones each individual and resets its fitness. Lastly it loops through each individual and passes it to the \emph{mutateIndividual} method.

The mutateIndividual method loops through the individual's genome by index. It uses this index to acquire each gene's corresponding parameters from the species class (e.g. minimum, maximum, mutation probability, favour mutated, favour byte etc.). Using the thread's random number generator stored in the EvolutionState object is decides whether to mutate by calling {\ttfamily nextBoolean(mutationProbability)}. If so it mutates the gene by favour or reset mutation (again decided by the gene parameters). For reset mutation it replaces the gene with a random byte between the gene's minimum and maximum. For favour mutation it makes another call to nextBoolean, with the favour probability parameter. If this returns true, the gene is replaced by the favour byte, otherwise a byte between the minimum and maximum is chosen, excluding the favour byte. The code for this method is given below (edited for brevity).

\begin{minipage}{0.9\linewidth}
\centering
\begin{lstlisting}[language=scala]
protected def mutateIndividual(state: EvolutionState, thread: Int, vecInd: ByteVectorIndividual, species: RulesetSpecies): ByteVectorIndividual = {
    for (n <- 0 until vecInd.genome.length) {
        if (state.random(thread).nextBoolean(species.mutationProbability(n))) {
            if (species.favourMutation(n)) {
                if (random.nextBoolean(favourProbability))
                    vecInd.genome(n) = species.favourByte(n)
                else
                    vecInd.genome(n) = getRandomByte(
                       species.minGene(n).toByte,
                       species.maxGene(n).toByte,
                       state.random(thread),
                       species.favourByte(n)
                     )
\end{lstlisting}
\end{minipage}
                    
\begin{minipage}{0.9\linewidth}
\centering
\begin{lstlisting}[language=scala]
            } else {
                vecInd.genome(n) = getRandomByte(
                  species.minGene(n).toByte,
                  species.maxGene(n).toByte,
                  state.random(thread)
                )    
            }
        }
    }
\end{lstlisting}
\end{minipage}
                    
\begin{minipage}{0.9\linewidth}
\centering
\begin{lstlisting}[language=scala]
    vecInd.evaluated = false
    vecInd
}
\end{lstlisting}
\end{minipage}

\subsubsection{Evaluation}

The Evaluator class prototypes the AgentRulesetEvaluator class and holds one clone for each evaluation thread. The AgentRulesetEvaluator overrides three methods from the Problem class: \emph{setup} which is called before creating the prototype, \emph{prepareToEvaluate} which is called once per thread clone before evaluation, and \emph{evaluate} which is called on single individuals multiple times per thread.

The setup method loads in the MWLevelOptions, MWEvaluationMultipliers, update lambda, number of levels, generational level seeds and the default ruleset action using the EvaluationParamUtil class. As these are built into the prototype they are shared across all evaluation threads, taking advantage of their immutability (explained in section /ref{subsec:paramclasses}). 

The MWEvaluationTask instance is built with these options once per thread, in the prepareToEvaluate method. As the game loop is mutating process it cannot be shared across threads, but as the task can be reset and new agents injecting in in can be used multiple times in one thread.

The evaluate method is passed an individual, which is verified to be a ByteVectorIndividual. The individual's genome is built into a ruleset, which is used to initialise an MWRulesetAgent. This agent, along with the level seed for the current generation, is injecting into the task. The task then evaluates the agent, returning a score, which is attached to the individual as its fitness.


\begin{minipage}{0.9\linewidth}
\centering
\begin{lstlisting}[language=scala]
override def evaluate(state: EvolutionState, individual: Individual, subpop: Int, thread: Int): Unit = {
    individual match {
        case ind: ByteVectorIndividual => {
            if (task.isDefined) {
                val evalTask = task.get
                val name = this.buildIndAgentName(state, individual, subpop, thread)
                val ruleset: Ruleset = Ruleset.buildFromArray(ind.genome, defaultAction)
                val agent: Agent = MWRulesetAgent(name, ruleset)
            
                val iFitness = evalTask.withAgent(agent)
                                      .withLevelSeed(_taskSeeds(state.generation))
                                      .evaluate
\end{lstlisting}
\end{minipage}

\begin{minipage}{0.9\linewidth}
\centering
\begin{lstlisting}[language=scala]
                ind.fitness match {
                    case _: SimpleFitness =>  {
                        ind.fitness.asInstanceOf[SimpleFitness].setFitness(state, iFitness.toDouble, false)
                        ind.evaluated = true
                    }
\end{lstlisting}
\end{minipage}

\begin{minipage}{0.9\linewidth}
\centering
\begin{lstlisting}[language=scala]
                    case _ => {
                         state.output.fatal("This evaluator (EvolvedAgentRulesetEvaluator) requires a individuals to have SimpleFitness")
                    }
                }
            } else {
                state.output.fatal("Task was not defined when evaluating individual, implying prepareToEvaluate was not run on this instance.")
            }
        }
        case _ => {
          state.output.fatal("This evaluator (AgentRulesetEvaluator) requires a ByteVectorIndividual")
        }
    }
}

\end{lstlisting}
\end{minipage}

\subsubsection{Statistics}

The Statistics class and its subclasses are setup at the beginning of the learning run and then called as `hook' at various points during the generational loop. RulesetEvolveStatistics implements \emph{setup} and two of the hooks, \emph{postEvaluationStatistics} and \emph{finalStatistics}. The setup method reads the filenames for the generation log and the final agent exports. It registers the log file with EvolutionState's \emph{Output} instance to allow it to be called from the other methods. 

The postEvaluationStatistics method is passed the EvolutionState, from which it can access the current population and generation number. It loops through the individuals of the current population, calculating the average fitness and determining the best individual. The average fitness, best fitness, current level seed, generation number and the best individuals genome are all written to the generation log using the Output class:

\begin{minipage}{0.9\linewidth}
\centering
\begin{lstlisting}[language=scala]
state.output.println(
val all = "~all~ " + genNum + "," + levelSeed + "," + avScore + "," + bestScore

"--------- GENERATION " + genNum + " ---------\n", genLog)
state.output.println(all + 
        "\nLevel Seed    : " + levelSeed + 
        "\nAverage Score : " + avScore + 
        "\nBest Score    : " + bestScore + 
        "\nBest Agent    :-" +
        "\n    " + agentStr + 
        "\n-----------------------------------\n\n", genLog)
\end{lstlisting}
\end{minipage}
 
The best individual in the generation is saved to the currentBestIndividual field. Further checks to see if it is the best overall individual or has the biggest difference to the average are also performed, resulting in the individual being saved to the bestOverallIndividual and biggestDiffIndividual fields respectively. These checks are also controlled by generation number, only being performed after a certain generation, specified in the parameter file.

The finalStatistics method retrieves the current\-Best\-Individual, overall\-Best\-Individual and biggest\-Diff\-Individual. From each it constructs a ruleset, initialises an agent with the correct default action and passes it to agent IO utility class to be persisted with the corresponding filenames captured during setup.


%-----------------------------------------------------------------------
% Testing
%-----------------------------------------------------------------------

\subsection{Testing}

ECJ's structure makes unit testing extremely difficult, due mainly to its heavy reliance on the EvolutionState singleton. It is utilised by every class and passed to nearly every method. ScalaMock is unable to mock this object as it is not accessed through an interface and building a fixed instance would be overly time consuming (and tantamount to simply running the software).

Ultimately, besides the EvaluationParamsUtil class, no testing was performed. Instead, the implementation adopted a `fail loudly' approach, where any discrepancy was logged and triggered a program shutdown. In this way errors were easily noticed and corrected.

The lack of testing is not desirable, and given more time, efforts would have been made to rectify this. However, as the software has no external `users' (it being used as a tool solely by those that wrote it) it did not affect the final result.


%-----------------------------------------------------------------------
% Running
%-----------------------------------------------------------------------

\subsection{Running}

As learning could take several hours, or even days, the software was run on an external quad-core server. Maven was used to package the main project, with all its dependencies in to a jar file, which allowed it to easily transferred and run. As the project went through several revisions and performed many learning runs, shell scripts were written to facilitate this process. The push script called Maven to package the project, running all tests, which was then copied to the server (using ssh), along with parameter files and their runner scripts. These runner scripts were then run remotely to begin the learning process. On completion another script retrieved the generation log and the agent files.

Immutability and thread safety were a major focus of the project. This was done to allow the evaluation and breeding stages to be run over four threads (to take advantage of the server's quad-core). However, during the initial run it was quickly discovered that multithreading the evaluation process was impossible. Even though each thread had its own evaluation task instance, many of the game engine's assets were held statically and were mutable. Moreover, the MarioEnvironment was implemented as a singleton. This meant only one task could be performed at a time, resulting in the evaluation stage being set to run in only one thread. Despite this setback, learning runs took approximately 6 hours to complete.

%-----------------------------------------------------------------------
% Parameter Revisions
%-----------------------------------------------------------------------

\subsection{Parameters and Revisions}
\label{subsec:learnparam}

\begin{table}
  \begin{adjustwidth}{-0cm}{-0cm}
  \begin{center} \footnotesize
    \begin{tabular}{ | c | c | c | c | c | c | l |}
    \hline
    \textbf{Level} & \textbf{Difficulty} & \textbf{Enemies} & \textbf{Type} & \textbf{Length} & \textbf{Time} & \textbf{Other} \TBstrut \\ \thickhline
    \textbf{1} & 2 & N & 1 & 200 & 100 &  \\ \hline
    \textbf{2} & 3 & N & 0 & 200 & 100 & \\ \hline
    \textbf{3} & 5 & N & 2 & 200 & 100 & \\ \hline
    \textbf{4} & 10 & N & 0 & 200 & 100 & \\ \hline
    \textbf{5} & 2 & N & 0 & 200 & 100 & Flat Level \\ \hline
    \textbf{6} & 7 & N & 0 & 200 & 100 & Flat Level \\ \hline
    \textbf{7} & 2 & Y & 0 & 200 & 100 & Frozen Enemies \\ \hline
    \textbf{8} & 2 & Y & 0 & 200 & 100 &  \\ \hline
    \textbf{9} & 3 & Y & 1 & 200 & 100 & Tubes \\ \hline
    \textbf{10} & 5 & Y & 0 & 200 & 100 & Cannons, no Blocks \\ \hline
    \end{tabular}
  \end{center}
  \end{adjustwidth}
  \caption{Level playing parameters of the final learning run. Type 0 is outside, and type 1 and 2 are inside. }
  \label{tab:evalparams}
\end{table}

\begin{table}
  \parbox{.45\linewidth}{
  \begin{center} \small
    \begin{tabular}{ | l | c | }
    \hline
    \textbf{Statistic} & \textbf{Multiplier} \TBstrut \\ \thickhline
    Distance & 1 \\ \hline
    Completion & 3200 \\ \hline
    Mario Mode & 200 \\ \hline
    Enemy Kills & 100 \\ \hline
    Time Left & 2 \\ \hline
    \end{tabular}
   \end{center}

  \caption{Evaluation multipliers of the final learning run.}
  \label{tab:multparams}
  }
  \hfill
  \parbox{.5\linewidth}{
  \begin{center} \small
    \begin{tabular}{ | l | c | }
    \hline
    \textbf{Gene Parameter} & \textbf{Probability} \TBstrut \\ \thickhline
    Condition Mutation & 0.05 \\ \hline
    Condition Favour & 0.5 \\ \hline
    MarioMode Favour & 0.95 \\ \hline
    EnemyLeft Favour & 0.9 \\ \hline
    Action Mutation & 0.09 \\ \hline
    \end{tabular}
  \end{center}
  \caption{Mutation parameters of the final learning run.}
  \label{tab:mutaparams}
}
\end{table}

Many learning runs were performed during the project, including numerous parameter revisions. For brevity, only the most significants changes will be presented.

Initial runs used parameter files that closely followed the approach used by the REALM agent team. Conditions favoured the {\footnotesize DONT\_CARE} byte with a probability of $0.4$, whereas actions were reset mutated. Every gene had a mutation probability of 0.2. Agents were evaluated over 10 levels in a similar style to REALM's method. The levels varied over a wide range of difficulties, but favoured easier levels. Enemies were enabled for all but 2 levels. The evaluation multipliers rewarded distance, level completion, Mario's final mode, kills and time left upon completion.

The agents produced with these parameters did not display any interesting behaviours and functioned similarly to the handcrafted Forward Jumping agent (\ref{subsec:hca}). the first revision aimed to address this.

Mutation probability of both conditions and actions was lowered and the {\footnotesize DONT\_CARE} condition favour probability was increased to 0.5. The biggest change came with the evaluation task. Agents now played a narrower band of difficulties, generally more difficult than the first set. Enemies were only enabled for the last 4 levels and pits were enabled for all. These changes were made as they suited a reactive agent over the blind forward jumping approach. Lastly, the level completion was given more weight.

Agents evolved with these parameter demonstrated much more compelling behaviours, as well as higher overall attainment. However, it was observed that much of their ruleset's were unused.

A second revision was made with an aim to increasing the number of rules used and to reduce the search space. Favour probability for genes corresponding to conditions on MarioMode and EnemyLeft were greatly increased, making {\footnotesize DONT\_CARE} likely in these conditions on all rules.

The agents produced with this final approach are evaluate in depth on section \ref{subsec:evallearnt}. Tables \ref{tab:evalparams}, \ref{tab:multparams} and \ref{tab:mutaparams} summarise the final parameters, which can be found in full, in ECJ's file format, in the appendix (\ref{}). !APPENDIX!


%___________________________________________
%*************************************************************
% Evaluation
%___________________________________________
%^^^^^^^^^^^^^^^^^^^^^^^^^^^^^^^^^^^^^^^^^^^^^^^^^^^

\section{Evaluation}
\vspace{\baselineskip}



%-----------------------------------------------------------------------
% Agent Framework
%-----------------------------------------------------------------------

\subsection{Agent Framework}



%-----------------------------------------------------------------------
% Learnt Agents
%-----------------------------------------------------------------------

\subsection{Learnt Agents}
\label{subsec:evallearnt}

%-----------------------------------------------------------------------
% Learning
%-----------------------------------------------------------------------

\subsection{Learning Process}


%-----------------------------------------------------------------------
% Project Evaluation
%-----------------------------------------------------------------------

\subsection{Project Evaluation}

\clearpage
\begin{thebibliography}{11}

\bibitem{myangelcafe}
  Siyuan Xu,
  \emph{History of AI design in video games and its development in RTS games},
  Department of Interactive Media \& Game development, Worcester Polytechnic Institute, USA,
  \url{https://sites.google.com/site/myangelcafe/articles/history_ai}.

\bibitem{pacmanghosts}
  Chad Birch,
  \emph{Understanding Pac-Man Ghost Behaviour},
  \url{http://gameinternals.com/post/2072558330/understanding-pac-man-ghost-behavior},
  2010.
  
\bibitem{halflife}
  Alex J. Champandard,
  \emph{The AI From Half-Like's SDK in Retrospective},
  \url{http://aigamedev.com/open/article/halflife-sdk/},
  2008.
  
\bibitem{fear}
  Tommy Thompson,
  \emph{Facing Your Fear},
  \url{http://t2thompson.com/2014/03/02/facing-your-fear/},
  2014.

\bibitem{halo}
  Damian Isla,
  \emph{GDC 2005 Proceeding: Handling Complexity in the Halo 2 AI},
  \url{http://www.gamasutra.com/view/feature/130663/gdc_2005_proceeding_handling_.php},
  2005.
  
\bibitem{rome}
  Alex J. Champandard,
 \emph{Monte-Carlo Tree Search in TOTAL WAR: Rome II Campaign AI},
 \url{http://aigamedev.com/open/coverage/mcts-rome-ii/},
 2014.
 
\bibitem{playermod}
  Georgios N. Yannakakis, Pieter Spronck, Daniele Loiacono and Elisabeth Andre,
  \emph{Player Modelling},
  \url{http://yannakakis.net/wp-content/uploads/2013/08/pm_submitted_final.pdf},
  2013.
  
\bibitem{skyrim}
  Matt Bertz,
  \emph{The Technology Behind The Elder Scrolls V: Skyrim},
  \url{http://www.gameinformer.com/games/the_elder_scrolls_v_skyrim/b/xbox360/archive/2011/01/17/the-technology-behind-elder-scrolls-v-skyrim.aspx},
  2011.
  
\bibitem{overmind}
  \emph{The Berkeley Overmind Project},
  University of Berkeley, California.
  \url{http://overmind.cs.berkeley.edu/}.
  
\bibitem{cellz}
  Simon M. Lucas,
  \emph{Cellz: A simple dynamical game for testing evolutionary algorithms},
  Department of Computer Science, University of Essex, Colchester, Essex, UK,
  \url{http://citeseerx.ist.psu.edu/viewdoc/download?doi=10.1.1.96.5068&rep=rep1&type=pdf}. 
  
\bibitem{panorama}
  G. N. Yannakakis and J. Togelius,
  \emph{A Panorama of Artificial and Computational Intelligence in Games},
  IEEE Transactions on Computational Intelligence and AI in Games,
  \url{http://yannakakis.net/wp-content/uploads/2014/07/panorama_submitted.pdf},
  2014.
  
\bibitem{marioaicomp}
  Julian Togelius, Noor Shaker, Sergey Karakovskiy and Georgios N. Yannakakis,
  \emph{The Mario AI Championship 2009-2012},
  AI Magazine 34 (3), pp. 89-92,
  \url{http://noorshaker.com/docs/theMarioAI.pdf},
  2013.

\bibitem{suttonrl}
  Richard S. Sutton and Andrew G. Barto,
  \emph{Reinforcement Learning: An Introduction}
  The MIT Press, Cambridge, Massachusetts, London, England,
  Available: \url{http://webdocs.cs.ualberta.ca/~sutton/book/ebook/the-book.html},
  1998.

\bibitem{ev-comp} 
  Gary B. Fogel et al.,
  \emph{Evolutionary Programming}
  Scholarpedia, 6(4):1818.,
  \url{http://www.scholarpedia.org/article/Evolutionary_programming},
  2011.
  
\bibitem{modernai3}
  Stuart J. Russell, Peter Norvig,
  \emph{Artificial Intelligence: A Moden Approach (3rd ed.},
  Upper Saddle River, New Jersey,
  2009.
  
\bibitem{modernai1}
  Stuart J. Russell, Peter Norvig,
  \emph{Artificial Intelligence: A Moden Approach (1st ed.},
  Upper Saddle River, New Jersey,
  1995.
  
\bibitem{rbsys}
  Anoop Gupta, Charles Forgy, Allen Newell, and Robert Wedig,
  \emph{Parallel Algorithms and Architectures for Rule-Based Systems},
  Carnegie-Mellon University Pittsburgh, Pennsylvania,
  ISCA '86 Proceedings of the 13th annual international symposium on Computer architecture, pp.~28-37,
  1986.
  
\bibitem{mitchellga}
  Melanie Mitchell,
  \emph{An Introduction to Genetic Algorithms},
  Cambridge, MA: MIT Press,
  1996.

\bibitem{es-book}
  Hans-Georg Beyer and Hans-Paul Schwefel,
  \emph{Evolution Strategies: A Comprehensive Introduction},
  Natural Computing 1: 3?52,
  2002.

\bibitem{rlheli}
  Andrew Y. Ng, Adam Coates, Mark Diel, Varun Ganapathi, Jamie Schulte, Ben Tse, Eric Berger and Eric Liang,
  \emph{Inverted autonomous helicopter flight via reinforcement learning},
  International Symposium on Experimental Robotics,
  \url{http://www.robotics.stanford.edu/~ang/papers/iser04-invertedflight.pdf},
  2004.
  
\bibitem{rlhci}
  S. Singh, D. Litman, M. Kearns and M. Walker,
  \emph{Optimizing Dialogue Management with Reinforcement Learning: Experiments with the NJFun System},
  Journal of Artificial Intelligence Research (JAIR), Volume 16, pages 105-133, 2002,
  \url{http://web.eecs.umich.edu/~baveja/Papers/RLDSjair.pdf}.
  
\bibitem{evolutioningamedesign}
  Alex J. Champandard,
  \emph{Making Designers Obsolete? Evolution in Game Design},
  \url{http://aigamedev.com/open/interview/evolution-in-cityconquest/},
  2012.
  
\bibitem{blackandwhite}
  James Wexler,
  \emph{A look at the smarts behind Lionhead Studio's `Black and White' and where it can and will go in the future},
  University of Rochester, Rochester, NY 14627,
  \url{http://www.cs.rochester.edu/~brown/242/assts/termprojs/games.pdf},
  2002.
  
\bibitem{projectgothamracing}
  Thore Graepal (Ralf Herbrich, Mykel Kockenderfer, David Stern, Phil Trelford),
  \emph{Learning to Play: Machine Learning in Games},
  Applied Games Group, Microsoft Research Cambridge,
  \url{http://www.admin.cam.ac.uk/offices/research/documents/local/events/downloads/tm/06_ThoreGraepel.pdf}.
  
\bibitem{forza}
  \emph{Drivatar\texttrademark\ in Forza Motorsport},
  \url{http://research.microsoft.com/en-us/projects/drivatar/forza.aspx}.
  
\bibitem{2012the}
  Sergey Karakovskiy and Julian Togelius,
  \emph{Mario AI Benchmark and Competitions},
  \url{http://julian.togelius.com/Karakovskiy2012The.pdf},
  2012.
  
\bibitem{2010the}
  Julian Togelius, Sergey Karakovskiy and Robin Baumgarten,
  \emph{The 2009 Mario AI Competition},
  \url{http://julian.togelius.com/Togelius2010The.pdf},
  2010.
  
\bibitem{2014how}
  Julian Togelius,
  \emph{How to run a successful game-based AI competition},
  \url{http://julian.togelius.com/Togelius2014How.pdf},
  2014.
  
\bibitem{torcs}
  \emph{TORCS: The Open Racing Car Simulation},
  \url{http://torcs.sourceforge.net/}.

\bibitem{scrc}
  Daniele Loiacono, Pier Luca Lanzi, Julian Togelius, Enrique Onieva, David A. Pelta, Martin V. Butz, Thies D. L�nneker, Luigi Cardamone, Diego Perez, Yago S�ez, Mike Preuss, and Jan Quadflieg,
  \emph{The 2009 Simulated Car Racing Championship},
  IEEE Transactions on Computational Intelligence and AI in Games, VOL. 2, NO. 2,
  2010.
  
\bibitem{2kbot}
  \emph{The 2K BotPrize},
  \url{http://botprize.org/}
  
\bibitem{starcomp}
  Michael Buro, David Churchill,
  \emph{Real-Time Strategy Game Competitions},
  Association for the Advancement of Artificial Intelligence, AI Magazine, pp.~106-108,
  \url{https://skatgame.net/mburo/ps/aaai-competition-report-2012.pdf},
  2012.
  
\bibitem{imb}
  \emph{Infinite Mario Bros.},
  Created by Markus Perrson,
  \url{http://www.pcmariogames.com/infinite-mario.php}.
  
\bibitem{handa}
  H. Handa,
  \emph{Dimensionality reduction of scene and enemy information in mario},
  Proceedings of the IEEE Congress on Evolutionary Computation,  
  2011.
  
\bibitem{rossbagnell}
  S. Ross and J. A. Bagnell, 
  \emph{Efficient reductions for imitation learning},
  International Conference on Artificial Intelligence and Statistics (AISTATS),
  2010.
  
\bibitem{realm}
  Slawomir Bojarski and Clare Bates Congdon,
  \emph{REALM: A Rule-Based Evolutionary Computation Agent that Learns to Play Mario},
  2010 IEEE Conference on Computational Intelligence and Games (CIG '10) pp.~83-90.

\bibitem{gramev}
   D. Perez, M. Nicolau, M. O'Neill, and A. Brabazon,
   \emph{Evolving Behaviour Trees for the Mario AI Competition Using Grammatical Evolution},
   Proceedings of EvoApps, 2010, pp.~123-132.

\bibitem{feetsies}
  E. R. Speed, 
  \emph{Evolving a mario agent using cuckoo search and softmax heuristics},
  Proceedings of the IEEE Consumer Electronics Society's Games Innovations Conference (ICE-GIC), 2010,
  pp.~1-7.
  
\bibitem{emapf}
  Thomas Willer Sandberg,
  \emph{Evolutionary Multi-Agent Potential Field based AI approach for SSC scenarios in RTS games},
   University of Copenhagen,
   2011.
  
\bibitem{ecj}
  \emph{ECJ: A Java-based Evolutionary Computation Research System},
  \url{https://cs.gmu.edu/~eclab/projects/ecj/}.
  
\bibitem{ecj-manual}
  Sean Luke,
  \emph{The ECJ Owner's Manual, v23},
  \url{https://cs.gmu.edu/~eclab/projects/ecj/docs/manual/manual.pdf}
  George Mason University,
  2015.
  
\bibitem{ecj-tut3}
  Sean Luke,
  \emph{ECJ Tutorial 3: Build a Floating-Point Evolution Strategies Problem},
  \url{https://cs.gmu.edu/~eclab/projects/ecj/docs/tutorials/tutorial3/index.html}
  George Mason University.
  
\bibitem{jgap}
  \emph{JGAP: Java Genetic Algorithms Package},
  \url{http://jgap.sourceforge.net/}.

\bibitem{astar}
  Robin Baumgarten,
  \emph{A* Mario Agent},
  \url{https://github.com/jumoel/mario-astar-robinbaumgarten}.
  
\bibitem{agentsmith}
  Ryan Small,
  \emph{Agent Smith: a Real-Time Game-Playing Agent for Interactive Dynamic Games},
  GECCO `08, July 12-16, 2008, Atlanta, Georgia, USA.
  \url{http://www.cs.bham.ac.uk/~wbl/biblio/gecco2008/docs/p1839.pdf}.

\end{thebibliography}
\end{document}

























